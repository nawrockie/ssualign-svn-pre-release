\section{Installation}
\label{section:install}

\sft{ssu-align} is part of the larger \sft{infernal} software
package. The first step in installing \sft{ssu-align} is to install
\sft{infernal}. The following quick installation instructions are
copied from the \sft{infernal} user's guide, where you will also find
more detailed instructions on how to customize the installation. 
After \sft{infernal} is installed, an additional crucial step is
required before you can use \sft{ssu-align}. This step is explained
in the section below entitled ``An additional crucial step that is
required before you can use SSU-align''.

\subsection{Quick installation instructions}

Download the source tarball (\prog{infernal.tar.gz}) from 
\htmladdnormallink{ftp://selab.janelia.org/pub/software/infernal/}
                  {ftp://selab.janelia.org/pub/software/infernal/}
or \\
\htmladdnormallink{http://infernal.janelia.org}
                  {http://infernal.janelia.org}

Unpack the software:

\user{tar xvf infernal.tar.gz}

Go into the newly created top-level directory (named either
\prog{infernal}, or \prog{infernal-xx} where \prog{xx} is a release
number):

\user{cd infernal}

Configure for your system, and build the programs:

\user{./configure}\\
\user{make}

Run the automated testsuite. This is optional. All these tests should
pass:

\user{make check}

The \sft{infernal} programs are now in the \prog{src/}
subdirectory. The \prog{ssu-align-0.1} subdirectory includes the
\prog{ssu-align} PERL script and other important files as described in
``SSU-align files'' below. The \sft{infernal} user's guide is in the 
top-level \prog{infernal} directory. The man pages are in the
\prog{documentation/manpages} subdirectory. You can manually move or
copy all of these to appropriate locations if you want. 

\emph{IMPORTANT:}
Make sure the \sft{infernal} programs (\prog{cmalign}, 
\prog{cmbuild}, \prog{cmcalibrate}, \prog{cmemit}, \prog{cmscore},
\prog{cmsearch}, and \prog{cmstat}) in the \prog{src/} subdirectory
as well as the \sft{easel} programs in the \\ \prog{easel/miniapps/}
subdirectory are in your \$PATH.

Optionally, you can install the man pages and programs in system-wide
directories. If you are happy with the default (programs in
\prog{/usr/local/bin/} and man pages in \prog{/usr/local/man/man1}),
do:

\user{make install}

That's all.  More complete instructions, including how to
change the default installation directories for \prog{make install},
can be found in the \sft{infernal} user's guide which is in 
PDF format in the top-level \prog{infernal/} directory. 

\subsection{SSU-align files}
The \prog{ssu-align-0.1/} subdirectory of \prog{infernal/} will include
the \prog{ssu-align} PERL script, the user's guide (this file) in
\prog{sa-userguide.pdf}, and the \emph{parameters} file 
\prog{sa-0p1.params}. You will need to manually edit this file as
explained below. Additionally you will find three subdirectories
within \prog{ssu-align-0.1/}:

\begin{wideitem}

\item[\emprog{documentation}] this includes the \prog{manpages/} and
  \prog{userguide/} subdirectories with files used to build this
  guide.

\item[\emprog{seeds}] this includes the Stockholm formatted alignments
  used to build the three default SSU CM files. These were derived
  from the \db{crw} database \cite{CannoneGutell02} as described in
  section~\ref{section:chap9}. It also includes the three individual CM
  files built from each of the seeds, and a subdirectory
  \prog{ss-diagrams} which includes several dozen structure diagrams
  created by \prog{esl-ssudraw} as explained in the Tutorial.

\item[\emprog{tutorial}] this includes some files that are used in the
  Tutorial section.

\end{wideitem}

\subsection{An additional crucial step that is required before you
can run SSU-align}

The \prog{ssu-align} PERL script needs to know where the
\sft{infernal} version 1.01 \prog{cmsearch} and \prog{cmalign}, and
the \sft{easel} \prog{esl-sfetch} executables are on your file
system. This is achieved by explicitly placing the paths in the text
file \prog{sa-0p1.params}. Take a look at this file:

\begin{sreoutput}
# Replace these paths with the corresponding paths to the infernal
# 1.01 executables on your file system.
$cmsearch = "/groups/eddy/home/nawrockie/infernal-1.01/src/cmsearch";
$cmalign = "/groups/eddy/home/nawrockie/infernal-1.01/src/cmalign";
$esl_sfetch = "/groups/eddy/home/nawrockie/infernal-1.01/easel/miniapps/esl-sfetch";
1;
\end{sreoutput}

It is vital that you replace these paths with absolute paths (do not 
use of '\~' or ``..'' in the paths) to these executable files on your own
filesystem. For example, if you unpacked \prog{infernal.tar.gz} in the 
directory \prog{Users/Lucy/} to create
\prog{Users/Lucy/infernal-1.01/}, then your \prog{sa-0p1.params} file
should define \$cmsearch as: 

\prog{\$cmsearch = "/Users/Lucy/infernal-1.01/src/cmsearch";}


