\subsection{Merging multiple alignments together}

INFERNAL's \prog{cmalign} program is capable of merging
two alignments into one. The two alignments must have both been
created by \prog{cmalign} (and the same version of \prog{cmalign}, 1.0
or later), and must have been created using the same exact CM. This
ability is potentially useful for saving time when aligning a large
number of sequences if you have access to a compute cluster, as
described in the previous section (``Splitting up large alignment
jobs''), or if you want to merge an existing reference alignment with
a newly created one, which is demonstrated below. Combining a new
alignment with a reference one may be useful in downstream
phylogenetic analysis, for example, if you know the classification of
the sequences in the reference alignment.

Imagine we wanted to merge the bacterial sequences from the 
\prog{seed-15} alignment we created at the beginning of the tutorial
with a single sequence alignment of the commonly used reference
sequence \emph{E. coli} \db{genbank} accession J01695. 

Merging alignments only makes sense and saves time if you've already
computed at least one of the two alignments you want to merge. For
this example I've provided the two alignments in the \\
\prog{infernal-1.01/ssu-align-0.1/tutorial} directory:

\begin{description}
\item[\emprog{seed-15.bacteria.stk}]
  An alignment of the five bacterial sequences from the \prog{seed-15.fa}
  sequence file. The beginning of this tutorial steps through how to create this file.

\item[\emprog{ecoli.bacteria.stk}]
  An alignment of the single J01695 bacterial sequence. This was
  created by aligning the sequence with the default bacterial CM.
\end{description}

To merge these two alignments into a single alignment called
\prog{seed-15-ecoli.bacteria.stk}, create or move
into the directory \prog{infernal-1.01/ssu-align-0.1/my-tutorial}, and
execute: 

\user{../../src/cmalign --merge -o seed-15-coli.bacteria.stk \\
../seeds/bacteria-0p1.cm ../tutorial/seed-15.bacteria.stk \\ ../tutorial/ecoli.bacteria.stk}

The resulting alignment will be 100\% identical to the alignment
\prog{cmalign} would have created if it were used to align a single
sequence file that included the five seed-15 sequences and the J01695
sequence together.

