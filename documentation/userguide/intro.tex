\section{Introduction}

\software{ssu-align} identifies and aligns small subunit ribosomal RNA
(SSU rRNA) genes in sequence datasets. It uses models of the primary
sequence and secondary structure conservation of SSU rRNA as inferred
by comparative sequence analysis and confirmed by crystal structure
determination from the \db{comparative rna website} (\db{crw}) 
(\htmladdnormallink{http://www.rna.ccbb.utexas.edu/}{http://www.rna.ccbb.utexas.edu/})
\cite{CannoneGutell02}.

\subsection{Overview}
The \software{ssu-align} package contains six different
programs: \textbf{ssu-align}\footnote{Confusingly, SSU-ALIGN is the name of the
  software package as well as the name of one of the programs within
  the package. In this guide, when written in lowercase bold-faced
  type (\textbf{ssu-align}) it refers to the executable program, and
  when written in all capital letters (\sft{SSU-ALIGN}) it refers to
  the package.}, 
, \textbf{ssu-build}, \textbf{ssu-draw}, 
\textbf{ssu-mask}, \textbf{ssu-merge}, and \textbf{ssu-prep}. 

You can use \textbf{ssu-align} to create multiple alignments of SSU
sequences with profile probabilistic models called covariance models
(CMs). The package includes three SSU rRNA CMs, an archaeal model,
a bacterial model, and a eukaryotic (nuclear) model. These are the
default models used, but others, including truncated versions of the
default models that represent only a specific region of SSU (such as
the V4 hypervariable region), can be created using the
\textbf{ssu-build} program.

Given an input dataset of unaligned SSU sequences, \textbf{ssu-align}
proceeds through two stages to generate
alignments. In stage 1, each of the sequences is aligned to each of
the three models based on primary sequence conservation to obtain a
score for each sequence to each model.  The model that gives the
highest scoring alignment to each sequence is the \emph{best-matching
  model} for that sequence.  In stage 2, each sequence is aligned to
its best-matching model using both primary sequence and secondary
structure conservation information. The end result is a separate
multiple alignment for each model that was the best-matching model for
at least one sequence in the input dataset.

The \textbf{ssu-mask} program can then be used to removing unreliable
alignment columns. This step is important prior to using the
alignments as input to phylogenetic inference tools that rely heavily
on alignment accuracy.  Alignment columns are selected for removal
based on automatically calculated ``confidence estimates'' for the
aligned residues, which are the probabilities that each residue
belongs in each column of the alignment given the CM. This is
discussed in more detail in the section 3 of this guide.

The \textbf{ssu-draw} program can be used to create secondary
structure diagrams of the alignments created by 
\textbf{ssu-align}. Diagrams that display per-column alignment
statistics, such as information content and frequency of insertions
and deletions, as well as those that display individual aligned
sequences can be drawn.

The \textbf{ssu-prep} and \textbf{ssu-merge} programs allow users to
divide up large alignment jobs into sets of parallel
\textbf{ssu-align} jobs on clusters or multi-core computers.

\subsection{What is included in this guide}

Section~\ref{section:install} describes how to install the
package. Section~\ref{section:chap9} is a chapter from my (submitted)
Ph.D. thesis \cite{Nawrocki09b} which describes how the program works,
and how the seed alignments were derived from \db{crw}
\cite{CannoneGutell02}.  The tutorial in
section~\ref{section:tutorial} walks through examples of using
\textsc{ssu-align} to build multiple alignments, mask alignments,
create models of a specific region of SSU, draw structure diagrams for
alignments, and split up large alignment jobs into smaller jobs to run
in parallel on a cluster.  Section~\ref{section:output} describes the
format of all the output files \sft{ssu-align} creates. Finally, the
\sft{ssu-align} Manual page, which explains the command-line options of
the program, is included at the end of this guide.

\subsection{What is included in this package}

\sft{ssu-align} includes the \sft{infernal} software package, which is
written in C and is installed along with \sft{ssu-align} by following the directions in
the Installation section of this guide. The six \sft{ssu-align}
programs are \sft{perl} scripts. 
Also included are the three default (archaea, bacteria,
and eukarya) SSU models and the \emph{seed} alignments they were built
from. These alignments were based on SSU structures and alignments
from the Comparative RNA Website (CRW) \cite{CannoneGutell02}. A
broader discussion of these models and alignments and the specific
procedure used to create them from the CRW data is explained in
section~\ref{section:chap9}.

Installation of the \sft{perl} (Practical Extraction and Report Language,
Larry Wall) interpreter package version 5.0 or later is required to
run the \software{ssu-align} \sft{perl} scripts.

\subsection{What this package does not do}

\textsc{ssu-align} only creates alignments, masks them, and draws SSU
secondary structure diagrams; it does not infer trees from the
alignments it creates, nor does it classify sequences beyond reporting
which model in the input CM file they score highest to.

\subsection{Other useful references}

The \software{infernal} user's guide \cite{infernalguide} included in
this package supplements this user's guide. My Ph.D. thesis 
(\htmladdnormallink{http://selab.janelia.org/publications.html}{http://selab.janelia.org/publications.html})
introduces and describes my work on CM methods. It includes three chapters (7-9) dedicated to
SSU rRNA alignment; chapter 9 is included in this guide as
section~\ref{section:chap9}, but the other chapters may include some
relevant background information as well. 
Other useful references on CMs include
\cite{Eddy94,Eddy02b,NawrockiEddy07,Nawrocki09,KolbeEddy09}. In
addition, the \database{rfam} database 
(\htmladdnormallink{http://rfam.sanger.ac.uk/}{http://rfam.sanger.ac.uk/})
uses \software{infernal} to search for and align
structural RNAs using more than 1000 different CMs, so its
publications may be of interest as well
\cite{Griffiths-Jones03,Griffiths-Jones05,Gardner09}.

\subsection{How to cite SSU-ALIGN}

Please cite the \sft{infernal} software publication:
\cite{Nawrocki09}, if you find \sft{ssu-align} useful for work that
you publish. Additionally, because \sft{ssu-align}'s seed alignments were
derived from the \db{comparative rna website} we ask that you cite
that database as well: \cite{CannoneGutell02}. 
%Finally, if you'd like to, you
%can also cite my Ph.D. thesis \cite{Nawrocki09b} because it is
%currently the most relevant and comprehensive reference for the
%program.

\subsection{Future development}

This initial 0.1 release of \textsc{SSU-ALIGN} is a prototype. 
Compute time for alignment is roughly one second per full length SSU
sequence. We hope to speed up the program and add additional SSU
profiles in future releases. Bug reports and feature requests are
appreciated.
