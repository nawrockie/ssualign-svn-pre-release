\section{Basic tutorial: defining and aligning SSU sequences using \textsc{ssu-align}}

Here is a tutorial walk-through of a small project with
\software{ssu-align}. This tutorial shows how to use the program for
it's most basic and fundamental purpose, creating multiple
alignments of SSU rRNA sequences. 

\subsection{Files used in this tutorial}

In this tutorial we'll use the following files. The file names below
include the path from the top-level \prog{ssu-align-0.1} directory
created when the \prog{ssu-align-0.1.tar} tarball was untarred:

  \begin{sreitems}{}
  \item[\prog{./seeds/ssu5-0p1.cm}] A covariance model (CM) file that
    defines five SSU rRNA CMs: an archael model, a bacterial model, a
    choloroplast model, a eukaryotic model and a metazoan
    mitrochondria model. These are the five default models used by
    \textsc{ssu-align}. More information can be found on these models
    in section 3.
  \item[\prog{tutorial/rocks.fa}] SSU rRNA sequences from an environmental
    survey sequencing project of microbes living in the pore space of
    rocks in the Rocky Mountains by J.J. Walker and Norm Pace
    \cite{Walker07}. 
  \item[\prog{./sa-0p1.params}] A file containing paths to
    \textsc{infernal} executable files that \textsc{ssu-align} needs
    to run. \textbf{\emph{IMPORTANT:}} You will likely need to change
    these paths to point to where you've installed the \prog{cmsearch}
    and \prog{cmalign} programs (these are created in
    \prog{infernal-1.0/src/} after building \textsc{infernal} version
    1.0 with \user{sh ./configure; make;}) and the \prog{esl-sfetch}
    program (which is created in \\ 
    \prog{infernal-1.0/easel/miniapps/}
    after building \textsc{infernal} version 1.0).
  \end{sreitems}

Create a new directory that you can work in, and copy the three
files listed above there. I'll assume for the following examples that you've
installed the \prog{ssu-align} PERL script in your path; if not, you'll
need to give a complete path name to the script (e.g. something like
\newline
\prog{/usr/people/nawrocki/ssualign/src/ssu-align} 
instead of just \prog{ssu-align}).

\subsection{An example run of \textsc{ssu-align}}

The file \prog{rocks.fa} contains 588 SSU sequences
\cite{Walker07}. \textsc{ssu-align} is designed to create structural
alignments of SSU sequences from studies like this one. To run it,
execute the following command:

\user{ssu-align ssu5-0p1.cm rocks.fa rocks sa-0p1.params}\\

As you can see, the \textsc{ssu-align} script takes 4 command line
arguments. The first is a CM file. The second is the target sequence
file to search (this must be in FASTA format). The third is a string
that is used as the basis for naming output files. In this case, the
script will create a new directory called \prog{rocks} where all
output files will be created; \prog{rocks} will also be included as a
prefix to the names of all output files.

\newpage

The program will first print a header describing the program version
used, command used, current date, and some other information. Then it
will begin stage 1. You'll see the following information printed to
the screen:

\begin{sreoutput}
# ssu-align :: define and align SSU rRNA sequences
# SSU-ALIGN 0.1 (June 2009)
# Copyright (C) 2009 HHMI Janelia Farm Research Campus
# Freely distributed under the GNU General Public License (GPLv3)
# - - - - - - - - - - - - - - - - - - - - - - - - - - - - - - - - - - - -
# command: /groups/eddy/home/nawrockie/ssualign/ssu-align ssu5-0p1.cm rocks.fa rocks sa-0p1.params
# date:    Wed Jun 17 16:51:25 2009
#
# Stage 1: Determining SSU start/end positions and best matching models.
\end{sreoutput}

In stage 1, the program scans the input sequences with each of the
five models in the CM file \\ \prog{ssu5-0p1.cm}. This has two
purposes.  First, it classifies each sequence by determining which
model in the input CM file gives each sequence the highest primary
sequence-based profile HMM alignment score. Secondly, it
defines the start and end points of the SSU sequences based on those
alignments.

Stage 1 takes about 5 minutes on this dataset. When it finishes you'll
see: 

\begin{sreoutput}
# Stage 1: Determining SSU start/end positions and best matching models.
#
# output file name             description                                   
# ---------------------------  ----------------------------------------------
  rocks.tab                    locations/scores of hits defined by HMM(s)
  rocks.archaea.hits.list      list of sequences to align with archaea CM
  rocks.archaea.hits.fa             48 sequences to align with archaea CM
  rocks.bacteria.hits.list     list of sequences to align with bacteria CM
  rocks.bacteria.hits.fa           341 sequences to align with bacteria CM
  rocks.chloroplast.hits.list  list of sequences to align with chloroplast CM
  rocks.chloroplast.hits.fa        199 sequences to align with chloroplast CM
\end{sreoutput}

This lists and briefly describes the 7 new files the script created in
a newly created subdirectory of the current working dir called
\prog{rocks/}. The first file \prog{rocks.tab} is output from
\textsc{infernal}'s \prog{cmsearch} program. The other 6 files are
model-specific, two files for each model that was the best-matching model for
at least 1 sequence in the input target sequence file
\prog{rocks.fa}. The \prog{.hits.list} suffixed files contain a list
of the sequences that match best to the model, and the \prog{.hits.fa}
suffixed files are those actual sequences. There were no sequences
that best-matched the eukaryotic model or the metazoan mitochondria
model, so no model-specific files were created for those two models.

\begin{comment}
Each of these file types is explained in more detail in the
``Description of output files'' section, but for now we'll continue
following the output of our example \prog{ssu-align} run.
\end{comment}

The program will now proceed to stage 2, the alignment stage. This
stage serially progresses through each model that was the
best-matching model for at least 1 sequence and uses the model to
align best-matching sequences. The alignments are computed by scoring
a combination of both sequence and secondary structure conservation,
as opposed to the scoring in stage 1 which only used sequence
conservation. As the alignment to each model finishes, two new lines
of text, one for each of two newly created files, will appear on the
screen. For this example, alignment to all three models takes about 6
minutes. When it finishes you'll see:

\newpage

\begin{sreoutput}
#
# Stage 2: Aligning each sequence to it's best matching model.
#
# output file name             description
# ---------------------------  ----------------------------------------------
  rocks.archaea.stk            archaea alignment
  rocks.archaea.cmalign        archaea cmalign output
  rocks.bacteria.stk           bacteria alignment
  rocks.bacteria.cmalign       bacteria cmalign output
  rocks.chloroplast.stk        chloroplast alignment
  rocks.chloroplast.cmalign    chloroplast cmalign output
  rocks.scores                 list of CM/HMM scores for each sequence
  rocks.log                    log file (*this* text printed to stdout)
#
# All output files created in directory ./rocks/
#
# CPU time (search):     00:04:36
# CPU time (alignment):  00:06:19
# CPU time (total):      00:10:56
\end{sreoutput}

The newly created alignments are the \prog{.stk} suffixed files. These
were created by \textsc{infernal}'s \prog{cmalign} program. The
\prog{cmalign} output is in the \prog{.cmalign} suffixed files.  As in
stage 1, these files were created in the \prog{./rocks/} subdirectory.

\subsection{Description of alignments}

The section ``Description of output files'' contains more information
on \textsc{ssu-align} output files, but for now we'll focus only on the new
alignments.  Take a look at the archaeal alignment we just created in
\prog{rocks/rocks.archaea.stk}.

This alignment includes consensus secondary structure annotation and
is in \emph{Stockholm format}. 
Stockholm format, the native alignment format used by \software{hmmer} and
\software{infernal} and the \database{Pfam} and \database{Rfam}
databases, is documented in detail in the \software{Infernal} User's
Guide which is included in this distribution in
\prog{infernal/documentation/userguide.pdf}.

For now, what you need to know about the key features of the alignment file is:
\begin{itemize}

\item The alignment is in an interleaved format, like other
  common alignment file formats such as \software{clustalw}.
  Lines consist of a name, followed by an aligned sequence;
  the alignment is split into blocks separated by blank lines.

\item Gaps are indicated by the characters ., \_, -, or \verb+~+.
  Notice that the first few blocks of the alignment are 100\% gaps.
  This is because the sequences in \prog{rocks.fa} are not full length
  SSU sequences, but rather partial sequences obtained using PCR
  primers that target well conserved regions within the SSU
  molecule. In this alignment you'll have to scroll down to about line
  1300 before you see an aligned residue.

\item Special lines starting with {\small\verb+#=GR+} followed by a
  sequence name and then {\small\verb+POST+} contain posterior
  probabilities for each aligned residue for the sequence they
  correspond to. These are confidence estimates in the correctness of
  the alignment.  The POSTX. row indicates the ’tens’ place of the
  confidence estimate while POST.X row indicates the ’ones’ place. So
  the confidence estimate for a residue with 9 in the POSTX. row, and
  7 in the POST.X row to two significant digits 97%. This means that
  if you sampled alignments from the posterior distribution of all
  possible alignments of this sequence to the model, about 97% of the
  time that residue would appear in that of the alignment. One special
  case: if the posterior probability is very nearly 100% (it’s
  difficult to be more precise on the exact percentage due to
  numerical precision issues) the annotated posterior values will be
  ``*'' characters in both the tens and one places. These confidence
  estimates can be used to mask the alignment to remove columns with
  significant fractions of ambiguously aligned residues as demonstrated
  in the next section.

\item A special line starting with {\small\verb+#=GC SS_cons+}
  indicates the secondary structure consensus. Gap characters annotate
  unpaired (single-stranded) columns. Base pairs are indicated by any
  of the following pairs: \verb+<>+, \verb+()+, \verb+[]+, or
  \verb+[]+.

\item A special ``RF'' line starting with {\small\verb+#=GC RF+}
  indicates the consensus, or ReFerence, model. Gaps in the RF line
  are \emph{insert} columns, where at least 1 sequence has at least 1
  inserted residue between two consensus positions. Uppercase residues
  in the RF line are well conserved positions in the model; lowercase
  residues are less well conserved.
\end{itemize}

\begin{comment}
To convert the alignment to fasta format that includes gaps, you can use the
\prog{scripts/stk2aln\_fa.pl} script. 
\end{comment}

\subsection{Pruning alignments based on probabilistic confidence
  estimates}

If your goal is to use a phylogenetic inference program to build trees
from alignments created by \textsc{ssu-align}, you may want to mask
out columns of the alignment that may include misaligned residues
first, and then only run the inference on the remaining columns where
you're confident the alignment is correct. 

As mentioned above, \textsc{infernal}'s \prog{cmalign} program
can automatically calculate posterior probabilities that estimate the
level of ambiguity in the alignment of each residue. These can be
interpreted as how confident the model is in each aligned residue in
each sequence.  MORE HERE.

The \prog{esl-alimanip} program can be used to mask
\textsc{ssu-align} generated alignments based on these confidence estimates.

I recommend a three step masking process. The first step is to remove
all insert columns from the alignment. Insert columns include residues
that are literally \emph{not aligned}, but rather just inserted in
between the appropriate two consensus positions. (There's more
explanation of this in the ``Background'' section).  Since inserts are
not aligned they should not be included in a phylogenetic inference.
The second step is to create a mask of the consensus model based on
the confidence estimates in the alignment. The final step is to prune
the alignment based on that mask.

Here's an example of using \prog{esl-alimanip} for the three steps. 
First, remove all insert columns and save the new alignment to 
\prog{rocks.archaea.noins.stk}:

\prog{esl-alimanip -k -o rocks.archaea.noins.stk rocks.archaea.stk}

The \prog{-k} option to \prog{esl-alimanip}a tells it to remove all
columns of the alignment that are gaps in the \prog{\#\=GC RF}
annotation. In \textsc{ssu-align} generated alignments, gaps in the
\prog{\#\=GC RF} annotation indicate a column is an insert column that
does not correspond to a consensus model position.

Second, generate a mask based on the alignment confidence estimates
in the insert-removed alignment :

\prog{esl-alimanip --omask my.mask --p-rf --pfract 0.95 --pthresh 0.95 -o trash.stk rocks.archaea.noins.stk}

The following will print to the screen:

\begin{sreoutput}
Average posterior value:                            0.96881 (35909 non-gap residues)
Average posterior value in non-gap #=GC RF columns: 0.97108 (35818 non-gap RF residues)


883 of 1508 RF columns (0.586) pass threshold
\end{sreoutput}

And the mask file \prog{my.mask} is created. Take a look at this
file. It is a single line of text of \prog{0}s and \prog{1}s of length
1508. The length is 1508 because the consensus archaeal model (see
``Models'' section) is 1508 residues long. A \prog{0} at position x
indicates that consensus position x will excluded (pruned away) when the mask
is applied. A \prog{1} at position x indicates that consensus position
x will be included (not pruned away) when the mask is applied. 

In general the columns where the alignment confidence is high will be
included, and where it is low will be excluded. More specifically, a
column x is excluded if less than 0.95 fraction (95\%) of the sequences that
have a residue (non-gap) in column x have a confidence estimate of
0.95 or better for that residue. All other columns are included.

These thresholds can be changed using command-line options to 
\prog{esl-alimanip}. The fraction threshold can be changed to
\prog{<x>} with the \prog{--pfract <x>} and the confidence estimate
threshold can be changed to \prog{<y>} with the \prog{--pthresh <y>}
option.

Note that only non-gap residues are considered when the mask is
generated. This means if a consensus column only has a single residue
in it, and it has high confidence (say 0.98), then that column will be
be included in the mask (not be pruned away). This may be be
undesirable, depending on what type of downstream phylogenetic inference
you are going to do. Many phylogenetic inference programs treat gaps
as missing data, and for such programs you may want to exclude very gappy
columns when masking. 

The \prog{esl-alimanip} program can also take the gappiness of a
column into account when masking. To use the same confidence estimate
thresholds as the previous \prog{esl-alimanip} mask generation
(\prog{--pfract 0.95 --pthresh 0.95}) but \emph{also} exclude any column 
with more than 25\% gaps, execute the command:

\prog{esl-alimanip --omask my2.mask -g --gapthresh 0.25 --p-rf --pfract 0.95 --pthresh 0.95 -o trash.stk rocks.archaea.noins.stk}

\begin{sreoutput}
Average posterior value:                            0.97108 (35818 non-gap residues)
Average posterior value in non-gap #=GC RF columns: 0.97369 (33556 non-gap RF residues)


756 of 769 RF columns (0.983) pass threshold
\end{sreoutput}

The final line indicates that only 769 consensus positions had less
than 25\% gaps, of these 769, 756 passed the confidence estimate
thresholds. The new mask \prog{my2.mask} is still 1508 characters
long, but now contains only 756 \prog{1}s.

The final step is applying the mask. This is also done with
\prog{esl-alimanip}:

\prog{esl-alimanip -k --mask-rf my2.mask -o my2.masked.archaea.stk
  rocks.archaea.noins.stk}

The new alignment is saved in \prog{my2.masked.archaea.stk}.

TO DO: show how to put mask onto a SS diagram. But not sure which
section it goes in, the extended tutorial on drawing SS diagrams, or
here? 









