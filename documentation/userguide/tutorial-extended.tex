\section{Extended Tutorial}

This tutorial walks through some other important aspects of \textsc{ssu-align}.

\subsection{Files used in this tutorial}

The subdirectory \prog{/tutorial-extended} in the \software{ssu-align}
distribution contains the files used in the tutorial:

  \begin{sreitems}{}
  \item[\prog{ssu.default.1p0.cm}] A covariance model (CM) file that
    defines five SSU rRNA CMs: an archael model, a bacterial model, a
    choloroplast wmodel, a eukaryotic model and a metazoan
    mitrochondria model. These are the five default models used by
    textsc{SSUalign}.
  \item[\prog{rocks.fa}] SSU rRNA sequences from the an environmental
    survey sequencing project of microbes living in the pore space of
    rocks in the Rocky Mountains by J.J. Walker and Norm Pace
    \cite{Walker07}. 
  \end{sreitems}

Create a new directory that you can work in, and copy all the files in
\prog{tutorial-extended} there. I'll assume for the following examples
that you've installed the \software{ssu-align} programs in your path;
if not, you'll need to give a complete path name to the programs
(e.g. something like
\newline
\prog{/usr/people/nawrocki/ssualign/src/ssu-align} 
instead of just \prog{ssu-align}).

\subsection{Creating a truncated model of a specific region of SSU rRNA}

Some SSU rRNA sequencing studies target a specific region of the SSU
rRNA molecule using PCR primers at the boundaries of that region. For
such studies it is recommended to build a new CM that only models the
region of the molecule targetted by the study. In this section I'll
demonstrate how to create such a CM.

For this example imagine our study is only targetting bacterial SSU
rRNA. We will use the bacterial SSU seed alignment that is included
with \textsc{SSUalign} as a starting point for creating our new,
truncated CM. The first step is to determine what consensus positions
in the bacterial seed alignment's consensus structure the targetted
oregion corresponds to. The consensus structure of the bacterial seed
alignment is shown in the basic tutorial section on page X.

In practice, you'll have to manually find your primer site and
determine their positions on this consensus structure. Every hundredth
residue is numbered, and every tenth residue is marked with a tick
mark, which should help. 

Imagine for this example our 5' primer begins at consensus position
35 and ends at position 397. The next step is to create a truncated
bacterial seed alignment that only models between consensus positions
35 and 397. We can do this with the \prog{esl-alimanip} program using
the full bacterial seed alignment as input:

\user{esl-alimanip --start-rf 35 --end-rf 397 -o mybac.stk bacteria.0p1.stk}

This command creates a new alignment called \prog{mybac.stk} which
includes the subset of the columns from \prog{bacteria.0p1.stk} that lie
between consensus positions 35 and 397 inclusively.

The next step is to build a new model from this new alignment with
\textsc{infernal's} \prog{cmbuild} program: 

\user{cmbuild --enone --gapthresh 0.8 mybac.cm mybac.stk}

Note: the \prog{--enone} and \prog{--gapthresh 0.8} flags are
recommended best practice for \textsc{SSUalign}. The \prog{--enone}
flag tells the program to turn off entropy-weighting, a
parameterization technique used to make CM homology search more
sensitive \cite{Nawrocki07} but that seems slightly detrimental to CM
alignment accuracy with SSU rRNA models. The \prog{--gapthresh 0.8}
flag tells the program to define any column that has less than 80\%
gaps in the seed as a consensus column. Different values than 0.8
could be used here, but 0.8 seems to yield good performance for SSU
alignment, and it was also used to build the five default
\textsc{SSUalign} models.

Now you can begin using your new model \prog{mybac.cm} to align
SSU sequences with \textsc{SSUalign}. You have two options. 
You can either use your new model by itself as the only model in an
\prog{ssu-align} run, or you can combine it with other models prior to
running \prog{ssu-align}. We'll walk through an example of each, and
discuss appropriate uses for each case.

Using a single model is only recommended if you believe all the input
sequences will match well to that model. In this case, if we use
\prog{mybac.cm} as the only model we are assuming all of the input
sequences are bacterial SSU subsequences that map to the 35-397 region
of the consensus bacterial model. The file \prog{partial.fa} has 10
sequences, 8 of which are bacterial SSU sequences that map to this region
and 2 are archaeal SSU sequences that map to the homologous region in
the archaeal consensus model (roughly positions 25 to 380). Let's try
using \prog{mybac.cm} to align these 10 seqs:

\user{ssu-align -F mybac.cm partial.fa single 1p0.params}

Now let's examine the \prog{single.scores} file in the \prog{single/}
subdir:

\begin{sreoutput}
#                                                     best-matching model                second-best model  
#                                       -----------------------------------------------  -----------------  
#     idx  sequence name                model name   beg   end    CM sc   struct   HMM sc  model name   HMM sc  HMMdiff
# -------  ---------------------------  --------  ----  ----  -------  -------  -------  --------  -------  -------
        1  gi|146141790|gb|EF522294.1|  Bacteria     1   344   403.20   100.11   312.76  Bacteria     0.00   312.76
        2  gi|146141797|gb|EF522301.1|  Bacteria     1   348   463.37    69.69   402.70  Bacteria     0.00   402.70
        3  gi|146141805|gb|EF522309.1|  Bacteria     1   315   475.02    63.31   416.05  Bacteria     0.00   416.05
        4  gi|146141812|gb|EF522316.1|  Bacteria     1   339   480.57    68.35   423.81  Bacteria     0.00   423.81
        5  gi|146141828|gb|EF522332.1|  Bacteria     1   336   450.46    70.82   392.66  Bacteria     0.00   392.66
        6  gi|146141831|gb|EF522335.1|  Bacteria     1   332   380.82    94.56   297.11  Bacteria     0.00   297.11
        7  gi|146141832|gb|EF522336.1|  Bacteria     1   334   476.07    69.57   415.65  Bacteria     0.00   415.65
        8  gi|146141837|gb|EF522341.1|  Bacteria    11   363   338.11    83.56   285.87  Bacteria     0.00   285.87
        9  gi|146141762|gb|EF522266.1|  Bacteria     6   329   414.44    57.76   363.13  Bacteria     0.00   363.13
       10  gi|146141767|gb|EF522271.1|  Bacteria     6   301   424.54    49.14   361.39  Bacteria     0.00   361.39
\end{sreoutput}

\begin{comment}
\item Use \prog{mybac.cm} as the CM file when running
  \textsc{SSUalign}. This is recommended if you believe all of your
  SSU sequences are bacterial SSU sequences within the specified
  region (consensus positions 35 to 397). 

\item Use \prog{mybac.cm} as one of several models in a multi-model CM
  file when running \textsc{SSUalign}. You can create multi-model CM
  files by simply concatenating them together. For example you can add
  it as a sixth model to the default \textsc{SSUalign} 0.1 five model
  CM file \prog{sa.01.abcem.cm} with: \user{cat sa.01.abcem.cm
    mybac.cm > six.cm}. This is recommended if you think only some of
  your sequences will be within the specified region (consensus
  positions 35 to 397), while others might be full length bacterial
  sequences, or even archaeal or eukaryotic sequences.

\end{enumerate}

Now let's try aligning some sequences with the new model. The file
\prog{partial.fa} is a contrived set of sequences, most of which match
to the 35-397 region of bacterial SSU, but some which match to the
analogous region in archaeal SSU. Running \textsc{ssu-align} on this
dataset will help demonstrate some important points. First let's build
a new CM file with 2 CMs, our new \prog{mybac.cm} model and the default
Archaeal model. This is as simple as concatenating the two
single model CM files:

\prog{cat mybac.cm Archaeal.cm >> my2.cm}

Now use it align the sequences in \prog{partial.fa}:

\prog{cat mybac.cm Archaeal.cm >> my2.cm}
\end{comment}

%%%%%%%%%%%%%%%%%%%%%%%%%%%%%%%%%%%%%%%%%%%%%%%%%%%%%%%%%%%%
\section{Advanced tutorial: drawing SSU secondary structure diagrams}
%%%%%%%%%%%%%%%%%%%%%%%%%%%%%%%%%%%%%%%%%%%%%%%%%%%%%%%%%%%%
\section{Advanced tutorial: creating new SSU models}
%%%%%%%%%%%%%%%%%%%%%%%%%%%%%%%%%%%%%%%%%%%%%%%%%%%%%%%%%%%%
\section{Advanced tutorial: splitting up large alignment jobs}
%%%%%%%%%%%%%%%%%%%%%%%%%%%%%%%%%%%%%%%%%%%%%%%%%%%%%%%%%%%%
\section{Advanced tutorial: merging your alignments with existing
  reference alignments}
%%%%%%%%%%%%%%%%%%%%%%%%%%%%%%%%%%%%%%%%%%%%%%%%%%%%%%%%%%%%
\section{? Advanced tutorial: manipulating alignments with the
  \prog{esl-alimanip} program}
%%%%%%%%%%%%%%%%%%%%%%%%%%%%%%%%%%%%%%%%%%%%%%%%%%%%%%%%%%%%
\section{Benchmarking \textsc{SSU-align} alignment accuracy}
%%%%%%%%%%%%%%%%%%%%%%%%%%%%%%%%%%%%%%%%%%%%%%%%%%%%%%%%%%%%

