\section{Description of output files}
\label{section:output}

This section describes the content of the different output file
formats created by \prog{ssu-align}. The descriptions below refer to
the files created in the \prog{seed-15/} directory by the example run
at the beginning of the Tutorial section using the command:

\user{ssu-align ../seeds/ssu3-0p1.cm ../tutorial/seed-15.fa seed-15 ../sa-0p1.params}\\

As shown in the Tutorial, that command results in \prog{ssu-align}
printing the following information to the screen:

\begin{sreoutput}
# Stage 1: Determining SSU start/end positions and best matching models.
#
# output file name            description                                
# --------------------------  -------------------------------------------
  seed-15.tab                 locations/scores of hits defined by HMM(s)
  seed-15.archaea.hits.list   list of sequences to align with archaea CM
  seed-15.archaea.hits.fa           5 sequences to align with archaea CM
  seed-15.bacteria.hits.list  list of sequences to align with bacteria CM
  seed-15.bacteria.hits.fa          5 sequences to align with bacteria CM
  seed-15.eukarya.hits.list   list of sequences to align with eukarya CM
  seed-15.eukarya.hits.fa           5 sequences to align with eukarya CM
#
# Stage 2: Aligning each sequence to it's best matching model.
#
# output file name            description
# --------------------------  -------------------------------------------
  seed-15.archaea.stk         archaea alignment
  seed-15.archaea.cmalign     archaea cmalign output
  seed-15.bacteria.stk        bacteria alignment
  seed-15.bacteria.cmalign    bacteria cmalign output
  seed-15.eukarya.stk         eukarya alignment
  seed-15.eukarya.cmalign     eukarya cmalign output
  seed-15.scores              list of CM/HMM scores for each sequence
  seed-15.log                 log file (*this* text printed to stdout)
#
# All output files created in directory ./seed-15/
\end{sreoutput}

In this section I'll walk through the format of these output files,
which are created in \prog{seed-15/}, a subdirectory of the directory
where the \prog{ssu-align} command was issued.

\subsection{.tab suffixed files}

The \prog{.tab} files are created by \textsc{infernal}'s
\prog{cmsearch} program which is called internally by
\prog{ssu-align}. \prog{cmsearch} builds profile HMMs for each model
in the input CM file and scans each input sequence for high scoring
alignments to the HMM\. The \prog{.tab} file lists the locations and
scores of each of these alignments. Let's look at the header of the 
\prog{seed-15.tab} file created here:

\begin{sreoutput}
# command:    /groups/eddy/home/nawrockie/infernal-1.01/src/cmsearch --hmm-cW 1.5 --no-null3 --noalign -T -1 --tab seed-15/seed-15.tab --viterbi seed-15/ssu3-0p1.cm ../tutorial/seed-15.fa
# date:       Thu Aug 20 06:32:05 2009
# num seqs:   15
# dbsize(Mb): 0.040518
#
# Pre-search info for CM 1: archaea
#
# rnd  mod  alg  cfg   beta  bit sc cut
# ---  ---  ---  ---  -----  ----------
#   1  hmm  vit  loc      -       -1.00
#
\end{sreoutput}

The first four lines list the \prog{cmsearch} command executed from
within \prog{ssu-align}, the date it was executed, the number of
sequences in the sequence file being searched, and the number of
millions of residues (Mb) searched in that sequence file (double the
actual size of the sequences because both strands are searched.

Next, the pre-search information describing how the search will be
conducted is printed for the first model in the CM file: archaea. 
The \prog{rnd}, \prog{mod}, \prog{alg} and \prog{cfg} columns report
that the first and only round of searching will be conducted with a
profile \prog{hmm} using the \prog{vit}erbi algorithm for scoring
\prog{loc}al alignments to the model. The \prog{beta}
column is irrelevant when searching with an HMM. The \prog{bit sc cut}
column shows the minimum score threshold, alignments scoring above
\prog{-1.00} bits will be reported.

After the pre-search information comes four lines of column headings
followed by data lines reporting high scoring local alignments. One
line is printed per alignment, or \emph{hit} (it is possible to see
more than one hit per sequence). Remember these are scores against the
archaea model.

\begin{sreoutputtiny}
#                                                          target coord   query coord                         
#                                                ----------------------  ------------                         
# target name                                         start        stop  start   stop    bit sc   E-value  GC%
# ---------------------------------------------  ----------  ----------  -----  -----  --------  --------  ---
  00052::Halobacterium_sp.::AE005128                      1        1473      1   1508   2080.08         -   58
  00052::Halobacterium_sp.::AE005128                   1229        1217      1   1508      1.41         -   54
  00013::Methanobacterium_formicicum::M36508              1        1476      1   1508   2108.16         -   56
  00013::Methanobacterium_formicicum::M36508           1227         984      1   1508      1.80         -   59
  00004::Nanoarchaeum_equitans::AJ318041                  1         865      1   1508   1112.07         -   67
  00004::Nanoarchaeum_equitans::AJ318041                619         570      1   1508      1.00         -   64
  00115::Pyrococcus_furiosus::U20163|g643670            260         309      1   1508      1.00         -   62
  00115::Pyrococcus_furiosus::U20163|g643670            922           1      1   1508   1354.25         -   66
  00121::Thermococcus_celer::M21529                     202        1687      1   1508   2223.71         -   66
  00121::Thermococcus_celer::M21529                    1237        1188      1   1508      1.00         -   62
  00035::Bacteroides_fragilis::M61006|g143965            60        1533      1   1508    576.22         -   51
  01106::Bacillus_subtilis::K00637                       55        1548      1   1508    768.61         -   55
  01106::Bacillus_subtilis::K00637                      684         672      1   1508     -0.63         -   46
  00072::Chlamydia_trachomatis.::AE001345               123         853      1   1508    212.41         -   51
  01351::Mycoplasma_gallisepticum::M22441               110         120      1   1508      1.43         -   36
  01351::Mycoplasma_gallisepticum::M22441               872          69      1   1508     95.30         -   45
  00224::Rickettsia_prowazekii.::AJ235272               104        1590      1   1508    613.67         -   51
  00224::Rickettsia_prowazekii.::AJ235272               748         739      1   1508      0.79         -   40
  01223::Audouinella_hermannii.::AF026040              1079        1767      1   1508    222.16         -   54
  01223::Audouinella_hermannii.::AF026040               880         871      1   1508      0.79         -   40
  01240::Batrachospermum_gelatinosum.::AF026045        1077        1761      1   1508    219.14         -   54
  01240::Batrachospermum_gelatinosum.::AF026045         878         869      1   1508      0.79         -   40
  00220::Euplotes_aediculatus.::M14590                  710        1081      1   1508     96.64         -   48
  00220::Euplotes_aediculatus.::M14590                  508         499      1   1508      0.79         -   40
  00229::Oxytricha_granulifera.::AF164122               335         344      1   1508      0.79         -   40
  00229::Oxytricha_granulifera.::AF164122               138          31      1   1508     86.31         -   53
  01710::Oryza_sativa.::X00755                         1190        1883      1   1508    202.90         -   53
  01710::Oryza_sativa.::X00755                          989         980      1   1508      0.79         -   40
#
# Post-search info for CM 1: archaea
#
\end{sreoutputtiny}

Here's a brief description of each column:

\begin{wideitem}
\item[\emprog{target name}] the name of the target sequence.

\item[\emprog{start (target coord)}] first sequence residue in the alignment

\item[\emprog{end (target coord)}] final sequence residue in the alignment
\end{wideitem}

Note that under \prog{target coord} some of the sequences' \prog{start} position is
greather than their \prog{stop} positions. This occurs if the program
has determined the sequence in the target sequence file is a reverse
complemented SSU sequence. 

The \prog{start} and \prog{stop} columns under \prog{query coord} are
uninformative for HMM searches like these.
(\prog{cmsearch} will print informative numbers in
these columns when the CM is used for a search.) With HMM searches the
\prog{begin} column will always be \prog{1} and the \prog{end} column
will always be the final position of the model.

\begin{wideitem}

\item[\emprog{bit sc}] the bit score of the HMM alignment of
  target sequence residues \prog{begin} to \prog{end}.

\item[\emprog{E-value}] this will always read \prog{-}. It would
  include an E-value of the bit score if the model had been calibrated
  with \prog{cmcalibrate}. Currently, \prog{ssu-align} does not use
  calibrated models, mainly because they are most useful for
  identifying remotely homologous structural RNAs usingmodels of
  much smaller RNAs, such as transfer RNA. See the \textsc{infernal}
  user's guide for more information.

\item[\emprog{GC\%}] the percent of the residues in the aligned
  sequence that are either \prog{G} or \prog{C}. This is largely
  irrelevant for SSU rRNA, but is sometimes useful when using
  \prog{cmsearch} with smaller models.
\end{wideitem}

\subsection{.hits.list suffixed files}

The \prog{.hits.list} files are simple files created for each model
that list the sequences that were best-matches to that model, and as a
result were aligned to that model in stage 2. If a model has zero
sequences for which it is the best-matching model, this file will not
be created. The file contains no new information that is not in the
\prog{.tab} file and is only created as a convenience, because it is
cumbersome to determine which model gave the highest score to a
particular sequence in the \prog{.tab} file.  For model \emph{x}, the
\prog{hits.list} file contains four columns providing four pieces of
data for each sequence whose best-matching model was \emph{x}. For
example, take a look at the \prog{seed-15.archaea.hits.list}
file:

\begin{sreoutput}
# List of 5 subsequences to align to CM: archaea
# Created by ssu-align.pl.
#
# target name                                  start    stop     score
# ------------------------------------------  ------  ------  --------
  00052::Halobacterium_sp.::AE005128               1    1473   2080.08
  00013::Methanobacterium_formicicum::M36508       1    1476   2108.16
  00004::Nanoarchaeum_equitans::AJ318041           1     865   1112.07
  00115::Pyrococcus_furiosus::U20163|g643670     922       1   1354.25
  00121::Thermococcus_celer::M21529              202    1687   2223.71
\end{sreoutput}

The four columns correspond to those of the same name in the {.tab}
suffixed files as explained above: the \prog{target name},
\prog{start} and \prog{stop} (which correspond to the target sequence
coordinates), and \prog{score}, which is the primary sequence HMM
score assigned to this subsequence by model \emph{x}: the scores of
all other hits from this and other other models are less than
this score.

\subsection{.hits.fa suffixed files}
This is a fasta formatted sequence file containing the sequences
listed in the corresponding \prog{hits.list} file. For example, the
file \prog{output/output.archaea.hits.fa} contains the two sequences
listed in \prog{output/output.archaea.hits.list}. These sequences were
copied from the original fasta sequence file \prog{seed-10.fa} that
was used as input to \prog{ssu-align}. Only the residues from
positions \prog{start} to \prog{stop} (as listed in the
\prog{hits.list} file) were copied, so sometimes the sequences in the
\prog{hits.fa} file will be subsequences of those from the original
file.  The \prog{ssu-align} script uses the \prog{hits.fa} files it
creates as input to the \prog{cmalign} program, which it calls
internally to generate the alignments.

\subsection{.cmalign suffixed files}

The \prog{.cmalign} files are the standard output created by the
\textsc{infernal} program \prog{cmalign} which is called internally
during stage 2 of the \prog{ssu-align} script. There is one such file
created for each model that was the best-matching model to at least
one sequence in \prog{ssu-align}'s search stage. Take a look at the
file \prog{seed-15.archaea.cmalign}. The first part of the file
is the header section:

\begin{sreoutput}
# cmalign :: align sequences to an RNA CM
# INFERNAL 1.0 (January 2009)
# Copyright (C) 2009 HHMI Janelia Farm Research Campus
# Freely distributed under the GNU General Public License (GPLv3)
# - - - - - - - - - - - - - - - - - - - - - - - - - - - - - - - - - - - -
# command: /groups/eddy/home/nawrockie/infernal-1.01/src/cmalign -p --mxsize 4096 --sub -o seed-15/seed-15.archaea.stk seed-15/ssu3-0p1.archaea.m1.cm seed-15/seed-15.archaea.hits.fa
# date:    Thu Aug 20 06:32:20 2009
\end{sreoutput}

This section includes the program version used, the copyright
information, the command used to execute \prog{cmalign}, and the date
of execution.

Next comes information on alignment parameters used by the program:

\begin{sreoutput}
#
# cm name                    algorithm  config  sub  bands     tau
# -------------------------  ---------  ------  ---  -----  ------
# archaea                      opt acc  global  yes    hmm   1e-07
\end{sreoutput}

The \prog{cm name} column reports the name of the model used for
alignment. \prog{algorithm} gives the name of the algorithm, in this
case \prog{opt acc} stands for \emph{optimal accuracy}. This is a
dynamic programming algorithm that returns the maximal posterior
labelling of all emitted residues in the alignment \cite{Holmes98}.
This algorithm is similar to the CYK algorithm described in
\cite{Nawrocki09b}, but returns the alignment that maximizes the sum
of posterior probability labels on aligned residues instead of the
maximally scoring alignment. In practice, the CYK and optimally
accurate alignment are very often identical, and if not, they are
nearly identical.  The next two columns, \prog{config} and \prog{sub},
read \prog{global} and \prog{yes} respectively, which tells us the
program will first predict the start and end points of the alignment
to the model using an HMM (the \prog{sub yes} part) and then align the
region of the model that spans from start to end \emph{globally} to
the sequence (the \prog{global} part). In this case, \emph{global}
alignment means that the program is forced to align the full model
region from start to end to the sequence, e.g. it is \emph{not}
allowed to skip large parts of the model without large score penalties
as it would if \emph{local} alignment was being performed. The
\prog{bands} column tells us that bands (constraints) from an HMM
alignment will be used to accelerate alignment to the CM. This is
explained more in Chapter 8 of \cite{Nawrocki09b}. The \prog{tau} column
reports the probability loss allowed when computing the HMM bands. In
this case, \prog{1e-07} probability mass is allowed outside each
band. 

The next section includes per-sequence information on the alignment
that was created:

\begin{sreoutputtiny}
#                                                                 bit scores                           
#                                                             ------------------                       
# seq idx  seq name                                      len     total    struct  avg prob      elapsed
# -------  ------------------------------------------  -----  --------  --------  --------  -----------
        1  00052::Halobacterium_sp.::AE005128           1473   2295.11    266.03     1.000  00:00:01.36
        2  00013::Methanobacterium_formicicum::M36508   1476   2306.88    247.42     0.999  00:00:01.36
        3  00004::Nanoarchaeum_equitans::AJ318041        865   1203.43    119.11     0.999  00:00:00.64
        4  00115::Pyrococcus_furiosus::U20163|g643670    922   1415.01     92.83     0.997  00:00:00.69
        5  00121::Thermococcus_celer::M21529            1486   2309.87    174.98     0.998  00:00:01.38
\end{sreoutputtiny}

We'll go through each of these columns:

\begin{wideitem}
\item[\emprog{seq idx}] the index of the sequence in the file.

\item[\emprog{seq name}] the name of the sequence.

\item[\emprog{len}] length of the sequence; the full sequence is
  aligned, no trimming of ends is permitted, as it was in the search
  stage with \prog{cmsearch}.

\item[\emprog{total}] the bit score of the CM alignment. For more
  information, see section 5 of the \sft{infernal} User's Guide.

\item[\emprog{struct}] the secondary structure score component of the
  \prog{total} bit score. These are the added bits that are due solely
  to the modelling of the consensus secondary structure of the
  molecule by the CM\@. 
  
\item[\emprog{avg prob}] the average posterior labeling, or confidence
  estimate, of the aligned residues. The higher this value is the less
  ambiguous and more well-defined the alignment is. The highest this
  can possibly be is \prog{1.000}, which means very nearly 100\% of
  the probability mass of the alignment to the model is contained in
  the single, optimally accurate alignment that was reported by the
  program. In other words, the reported alignment receives a
  significantly higher score than any other alternative alignment. The
  program derives this value by evaluating the score of every possible
  alignment (consistent with the HMM bands) of the sequence to the
  model, and comparing the best, optimal score versus all of the
  rest. This is described in a bit more detail in section~\ref{section:chap9}.

\item[\emprog{elapsed}] the amount of actual time (wall time) it took
  the program to align this sequence. In general, less well defined
  alignments with lower \prog{avg prob} will take longer than more
  well-defined ones. This is because the HMM bands are usually tighter
  and act as stricter constraints to the CM alignment when the
  alignment is well defined. Tighter bands lead to quicker alignments
  because the number of possible alignments to the CM that must be
  considered is less.
\end{wideitem}

\subsection{.stk suffixed files}
The \prog{.stk} suffixed files are Stockholm-formatted alignment
files. These are the alignments generated by \prog{cmalign}. The
statistics in the \prog{.cmalign} suffixed files correspond to these
alignments. One alignment is created for each model that was the
best-matching model to at least one sequence in \prog{ssu-align}'s
search stage. An explanation of Stockholm alignments can be found at
the beginning of the Tutorial section.

\subsection{.scores suffixed files}

The \prog{.scores} file are meant to be a useful summary file for each
run of \prog{ssu-align}. They contain various statistics from each of
the other output files for every sequence in the original input
sequence file. 

Take a look at the file \prog{tutorial/output/output.scores}

\begin{sreoutputtinywide}
#                                                                        best-matching model                  second-best model 
#                                                         -------------------------------------------------  -------------------
#     idx  sequence name                                  model name   beg   end    CM sc   struct   HMM sc  model name   HMM sc  HMMdiff
# -------  ---------------------------------------------  ----------  ----  ----  -------  -------  -------  ----------  -------  -------
        1  00052::Halobacterium_sp.::AE005128             archaea        1  1473  2295.11   266.03  2080.08  bacteria     471.45  1608.63
        2  00013::Methanobacterium_formicicum::M36508     archaea        1  1476  2306.88   247.42  2108.16  bacteria     693.10  1415.06
        3  00004::Nanoarchaeum_equitans::AJ318041         archaea        1   865  1203.43   119.11  1112.07  bacteria     255.78   856.29
        4  00115::Pyrococcus_furiosus::U20163|g643670     archaea      922     1  1415.01    92.83  1354.25  bacteria     378.49   975.76
        5  00121::Thermococcus_celer::M21529              archaea      202  1687  2309.87   174.98  2223.71  bacteria     632.38  1591.33
        6  00035::Bacteroides_fragilis::M61006|g143965    bacteria       5  1537  2126.24   376.62  1797.77  archaea      576.22  1221.55
        7  01106::Bacillus_subtilis::K00637               bacteria       1  1552  2440.64   274.80  2212.50  archaea      768.61  1443.89
        8  00072::Chlamydia_trachomatis.::AE001345        bacteria       1   879  1202.69   176.90  1030.27  archaea      212.41   817.86
        9  01351::Mycoplasma_gallisepticum::M22441        bacteria     881     5  1143.43   206.27   940.15  -                 -        -
       10  00224::Rickettsia_prowazekii.::AJ235272        bacteria      93  1594  2263.65   272.56  2037.49  archaea      613.67  1423.82
       11  01223::Audouinella_hermannii.::AF026040        eukarya        1  1770  2672.59   216.62  2481.69  archaea      222.16  2259.53
       12  01240::Batrachospermum_gelatinosum.::AF026045  eukarya        1  1764  2653.58   211.44  2463.49  archaea      219.14  2244.35
       13  00220::Euplotes_aediculatus.::M14590           eukarya        1  1082  1204.38    75.02  1103.29  -                 -        -
       14  00229::Oxytricha_granulifera.::AF164122        eukarya      600     1   844.65    21.89   788.12  -                 -        -
       15  01710::Oryza_sativa.::X00755                   eukarya       75  1886  2684.77   141.04  2570.56  archaea      202.90  2367.66
\end{sreoutputtinywide}

There are four rows containing column headings prefixed with
\prog{\#}. Then there are 15 data rows, one for each sequence in the
input sequence file \prog{seed-15.fa}. Data rows are
separated into 12 columns:

\begin{wideitem}
\item[\emprog{idx}] the index of the sequence in the file.

\item[\emprog{sequence name}] the name of the sequence.
\end{wideitem}

The next 6 columns all describe the \emph{best-matching} model for the
sequence. This is the model that assigned the highest primary
sequence-based local profile HMM alignment score to the sequence.  If
no model aligned the sequence with a score higher than the minimum
threshold of $100$ bits then the
sequence was skipped and not aligned, and all these columns will read
\prog{-}. (Note: the minimum bit score threshold value can be changed
to \prog{<x>} using the \prog{ssu-align} command-line option \prog{-b
  <x>}, as explained in the Manual page at the end of this guide).

\begin{wideitem}
\item[\emprog{model name}] name of best-matching model.

\item[\emprog{beg}] first sequence residue index in the maximal
  scoring local HMM alignment to the best-matching model.

\item[\emprog{end}] final sequence residue index in the maximal
  scoring local HMM alignment to the best-matching model.

\item[\emprog{CM sc}] the CM bit score for the best-matching model
  assigned to the sequence from positions \prog{beg} to \prog{end}.

\item[\emprog{struct}] the number of extra bits included in the
  CM bit score that are dervied from the secondary structure component
  of the model.

\item[\emprog{HMM sc}] the HMM bit score for the local alignment of
  the best-matching model to the sequence from positions \prog{beg} to
  \prog{end}.
\end{wideitem}

Bit scores are explained more in section 5 of the \sft{infernal}
User's Guide. 

The next 2 columns describe the \prog{second-best model}. This is the
model that assigned the second-highest primary sequence-based local 
profile HMM alignment score to the sequence. If only one models score
exceeded the minimum of $100$ bits then these columns will each read
``\prog{-}''.

\begin{wideitem}
\item[\emprog{model name}] name of second-best-matching model.

\item[\emprog{HMM sc}] the HMM bit score for the local alignment of
  the second-best-matching model to the sequence. This alignment was
  not necessarily from \prog{beg} to \prog{end} (those were the
  coordinates of the alignment to the best-matching model). The
  sequence coordinates of the second-best model's alignment can be
  found in the file \prog{seed-15.tab}.
\end{wideitem}

The final column, \prog{HMMdiff}, reports the score difference between
the best-matching model HMM alignment and the second-best matching
model HMM alignment. This is included because it is an indication of
how clearly homologous the sequence is to the best-matching model
instead of the second-best-matching model. The higher this score
difference is the more obvious it is that the sequence falls within
the sequence diversity represented by the best-matching model.
Sequences that are phylogenetically novel and do not obviously match
any single model much better than any other one should have relatively
small score differences in this column.

\subsection{.log suffixed files}

The \prog{.log} files include the text reported to the screen
(standard output) by \prog{ssu-align}. These files serve as a
reference to remind the user how the \prog{ssu-align} script was run
(parameters, input file names, etc.). 



