\section{Description of output files}
\label{sec:output}

This section describes the content of the different output file
formats created by the various programs in the \textsc{ssu-align}
package. For demonstration, the descriptions below refer to the
specific files created during the tutorial in
section~\ref{sec:tutorial}, in the directory \prog{myseqs/}. 
Specifically, we'll go over the files created by the following four
commands from the tutorial:

\user{ssu-align seed-15.fa myseqs}

\user{ssu-mask myseqs} 

\user{ssu-draw myseqs}

\subsection{ssu-align output files}

As shown in the tutorial, when executed with the command above, \prog{ssu-align} 
prints information about the output files it creates to the screen:

\begin{sreoutput}
# Stage 1: Determining SSU start/end positions and best-matching models...
#
# output file name         description                                
# -----------------------  -------------------------------------------
  myseqs.tab               locations/scores of hits defined by HMM(s)
  myseqs.archaea.hitlist   list of sequences to align with archaea CM
  myseqs.archaea.fa              5 sequences to align with archaea CM
  myseqs.bacteria.hitlist  list of sequences to align with bacteria CM
  myseqs.bacteria.fa             5 sequences to align with bacteria CM
  myseqs.eukarya.hitlist   list of sequences to align with eukarya CM
  myseqs.eukarya.fa              5 sequences to align with eukarya CM
#
# Stage 2: Aligning each sequence to its best-matching model...
#
# output file name         description
# -----------------------  ---------------------------------------
  myseqs.archaea.stk       archaea alignment
  myseqs.archaea.cmalign   archaea cmalign output
  myseqs.archaea.ifile     archaea insert info
  myseqs.bacteria.stk      bacteria alignment
  myseqs.bacteria.cmalign  bacteria cmalign output
  myseqs.bacteria.ifile    bacteria insert info
  myseqs.eukarya.stk       eukarya alignment
  myseqs.eukarya.cmalign   eukarya cmalign output
  myseqs.eukarya.ifile     eukarya insert info
  myseqs.scores            list of CM/HMM scores for each sequence
\end{sreoutput}

I'll walk through the format of these output files:

\subsubsection{.tab suffixed files}

The \prog{.tab} files are created by \textsc{infernal}'s
\prog{cmsearch} program which is called internally by
\prog{ssu-align}. \prog{cmsearch} builds profile HMMs for each model
in the input CM file and scans each input sequence for high scoring
alignments to the HMM\@. The \prog{.tab} file lists the locations and
scores of each of these alignments. Let's look at the header of the 
\prog{seed-15.tab} file created here:

\begin{sreoutput}
# command:    ssu-cmsearch --hmm-cW 1.5 --no-null3 --noalign -T -1 --tab myseqs/myseqs.tab --viterbi 
/usr/local/share/ssu-align-0.1/ssu-align-0p1.cm seed-15.fa
# date:       Tue May 18 05:48:34 2010
# num seqs:   15
# dbsize(Mb): 0.040518
#
# Pre-search info for CM 1: archaea
#
# rnd  mod  alg  cfg   beta  bit sc cut
# ---  ---  ---  ---  -----  ----------
#   1  hmm  vit  loc      -       -1.00
#
\end{sreoutput}

The first four lines list the \prog{cmsearch} command executed from
within \prog{ssu-align}, the date it was executed, the number of
sequences in the sequence file being searched, and the number of
millions of nucleotides (Mb) searched in that sequence file (double the
actual size of the sequences because both strands are searched).

Next, the pre-search information describing how the search will be
conducted is printed for the first model in the CM file: \prog{archaea}. 
The \prog{rnd}, \prog{mod}, \prog{alg} and \prog{cfg} columns report
that the first and only round of searching will be conducted with a
profile \prog{hmm} using the \prog{vit}erbi algorithm for scoring
\prog{loc}al alignments to the model. The \prog{beta}
column is irrelevant when searching with an HMM\@. The \prog{bit sc cut}
column shows the minimum score threshold, alignments scoring above
\prog{-1.00} bits will be reported.

After the pre-search information comes four lines of column headings
followed by data lines reporting high scoring local alignments. One
line is printed per alignment, or \emph{hit} (it is possible to see
more than one hit per sequence). Remember these are scores against the
archaea model.

\begin{sreoutputtiny}
# CM: archaea
#                                                                      target coord   query coord                         
#                                                            ----------------------  ------------                         
# model name  target name                                         start        stop  start   stop    bit sc   E-value  GC%
# ----------  ---------------------------------------------  ----------  ----------  -----  -----  --------  --------  ---
  archaea     00052::Halobacterium_sp.::AE005128                      1        1473      1   1508   2080.08         -   58
  archaea     00052::Halobacterium_sp.::AE005128                   1229        1217      1   1508      1.41         -   54
  archaea     00013::Methanobacterium_formicicum::M36508              1        1476      1   1508   2108.16         -   56
  archaea     00013::Methanobacterium_formicicum::M36508           1227         984      1   1508      1.80         -   59
  archaea     00004::Nanoarchaeum_equitans::AJ318041                  1         865      1   1508   1112.07         -   67
  archaea     00004::Nanoarchaeum_equitans::AJ318041                619         570      1   1508      1.00         -   64
  archaea     00121::Thermococcus_celer::M21529                     202        1687      1   1508   2223.71         -   66
  archaea     00121::Thermococcus_celer::M21529                    1237        1188      1   1508      1.00         -   62
  archaea     00115::Pyrococcus_furiosus::U20163|g643670            260         309      1   1508      1.00         -   62
  archaea     00115::Pyrococcus_furiosus::U20163|g643670            922           1      1   1508   1354.25         -   66
  archaea     00035::Bacteroides_fragilis::M61006|g143965            60        1533      1   1508    576.22         -   51
  archaea     01106::Bacillus_subtilis::K00637                       55        1548      1   1508    768.61         -   55
  archaea     01106::Bacillus_subtilis::K00637                      684         672      1   1508     -0.63         -   46
  archaea     00072::Chlamydia_trachomatis.::AE001345               123         853      1   1508    212.41         -   51
  archaea     01351::Mycoplasma_gallisepticum::M22441               110         120      1   1508      1.43         -   36
  archaea     01351::Mycoplasma_gallisepticum::M22441               872          69      1   1508     95.30         -   45
  archaea     00224::Rickettsia_prowazekii.::AJ235272               104        1590      1   1508    613.67         -   51
  archaea     00224::Rickettsia_prowazekii.::AJ235272               748         739      1   1508      0.79         -   40
  archaea     01223::Audouinella_hermannii.::AF026040              1079        1767      1   1508    222.16         -   54
  archaea     01223::Audouinella_hermannii.::AF026040               880         871      1   1508      0.79         -   40
  archaea     01240::Batrachospermum_gelatinosum.::AF026045        1077        1761      1   1508    219.14         -   54
  archaea     01240::Batrachospermum_gelatinosum.::AF026045         878         869      1   1508      0.79         -   40
  archaea     00220::Euplotes_aediculatus.::M14590                  710        1081      1   1508     96.64         -   48
  archaea     00220::Euplotes_aediculatus.::M14590                  508         499      1   1508      0.79         -   40
  archaea     00229::Oxytricha_granulifera.::AF164122               335         344      1   1508      0.79         -   40
  archaea     00229::Oxytricha_granulifera.::AF164122               138          31      1   1508     86.31         -   53
  archaea     01710::Oryza_sativa.::X00755                         1190        1883      1   1508    202.90         -   53
  archaea     01710::Oryza_sativa.::X00755                          989         980      1   1508      0.79         -   40
\end{sreoutputtiny}

Here's a brief description of each column:

\begin{wideitem}
\item[\emprog{model name}] the name of the model used for the search.

\item[\emprog{target name}] the name of the target sequence.

\item[\emprog{start (target coord)}] first sequence nucleotide in the alignment

\item[\emprog{end (target coord)}] final sequence nucleotide in the alignment
\end{wideitem}

Note that under \prog{target coord} some of the sequences' \prog{start} position is
greater than their \prog{stop} positions. This occurs if the program
has determined the sequence in the target sequence file is a reverse
complemented SSU sequence. 

The \prog{start} and \prog{stop} columns under \prog{query coord} are
uninformative for HMM searches like these.
(\prog{cmsearch} will print informative numbers in
these columns when the CM is used for a search.) With HMM searches the
\prog{begin} column will always be \prog{1} and the \prog{end} column
will always be the final position of the model.

\begin{wideitem}

\item[\emprog{bit sc}] the bit score of the HMM alignment of
  target sequence nucleotides \prog{begin} to \prog{end}.

\item[\emprog{E-value}] this will always read \prog{-}. It would
  include an E-value of the bit score if the model had been calibrated
  with \prog{cmcalibrate}. Currently, \prog{ssu-align} does not use
  calibrated models, mainly because they are most useful for
  identifying remotely homologous structural RNAs using models of
  much smaller RNAs, such as transfer RNA. See the \textsc{infernal} 
  user's guide \cite{Nawrocki09} for more information.

\item[\emprog{GC\%}] the percent of the nucleotides in the aligned
  sequence that are either \prog{G} or \prog{C}. This is largely
  irrelevant for SSU rRNA, but is sometimes useful when using
  \prog{cmsearch} with smaller models.
\end{wideitem}

\subsubsection{.hitlist suffixed files}

The \prog{.hitlist} files are simple files created for each model
that list the sequences that were best-matches to that model, and as a
result were aligned to that model in stage 2. If a model has zero
sequences for which it is the best-matching model, this file will not
be created for that model. The file contains no new information that is not in the
\prog{.tab} file and is only created as a convenience, because it is
cumbersome to determine which model gave the highest score to a
particular sequence in the \prog{.tab} file.  For model \emph{x}, the
\prog{hitlist} file contains four columns providing four pieces of
data for each sequence whose best-matching model was \emph{x}. For
example, take a look at the \prog{myseq.archaea.hitlist}
file:

\begin{sreoutput}
# List of 5 subsequences to align to CM: archaea
# Created by ssu-align.
#
# target name                                  start    stop     score
# ------------------------------------------  ------  ------  --------
  00052::Halobacterium_sp.::AE005128               1    1473   2080.08
  00013::Methanobacterium_formicicum::M36508       1    1476   2108.16
  00004::Nanoarchaeum_equitans::AJ318041           1     865   1112.07
  00121::Thermococcus_celer::M21529              202    1687   2223.71
  00115::Pyrococcus_furiosus::U20163|g643670     922       1   1354.25
\end{sreoutput}

The four columns correspond to those of the same name in the \prog{.tab}
suffixed files as explained above: the \prog{target name},
\prog{start} and \prog{stop} (which correspond to the target sequence
coordinates), and \prog{score}, which is the primary sequence HMM
score assigned to this subsequence by model \emph{x}: the scores of
all other hits from this and other other models are less than
this score.

\subsubsection{.fa suffixed files}
This is a FASTA-formatted sequence file containing the sequences
listed in the corresponding \prog{hitlist} file. For example, the
file \prog{myseqs/myseqs.archaea.fa} contains the five sequences
listed in \\
\prog{myseqs/myseqs.archaea.hitlist}. These sequences were
copied from the original sequence file \prog{seed-15.fa} that
was used as input to \prog{ssu-align}. Only the nucleotides from
positions \prog{start} to \prog{stop} as listed in the
\prog{hitlist} file were copied, so sometimes the sequences in the
\prog{.fa} file will be subsequences of those from the original
file.  \prog{ssu-align} uses the \prog{.fa} files it
creates as input to the \prog{cmalign} program, which it calls
internally to generate its structurally annotated alignments.

\subsubsection{.cmalign suffixed files}

The \prog{.cmalign} files are the standard output created by the
\textsc{infernal} program \prog{cmalign} which is called internally
during the alignment stage of \prog{ssu-align}. There is one such file
created for each model that was the best-matching model to at least
one sequence in \prog{ssu-align}'s search stage. Take a look at the
beginning of the file \prog{myseqs.archaea.cmalign}:

\begin{sreoutputwide}
# command: ssu-cmalign --cm-name archaea --mxsize 4096 --no-null3 --sub --ifile myseqs/myseqs.archaea.ifile 
-o myseqs/myseqs.archaea.stk /usr/local/share/ssu-align-0.1/ssu-align-0p1.cm myseqs/myseqs.archaea.fa
# date:    Tue May 18 05:48:47 2010
#
# cm name                    algorithm  config  sub  bands     tau
# -------------------------  ---------  ------  ---  -----  ------
# archaea                      opt acc  global  yes    hmm   1e-07
\end{sreoutputwide}

This section includes the command used to execute \prog{cmalign}, the
date of execution, and information on the alignment parameters used by
the program.

The \prog{cm name} column reports the name of the model used for
alignment. \prog{algorithm} gives the name of the algorithm, in this
case \prog{opt acc} stands for \emph{optimal accuracy} \cite{Holmes98}
(also known as maximum expected accuracy). This algorithm is similar
to the CYK algorithm described in \cite{Nawrocki09b}, but returns the
alignment that maximizes the sum of posterior probability labels on
aligned nucleotides instead of the maximally scoring alignment. In
practice, for SSU the CYK and optimally accurate alignment are very often
identical, and if not, they are nearly identical.  The next two
columns, \prog{config} and \prog{sub}, read \prog{global} and
\prog{yes} respectively, which tells us the program will first predict
the start and end points of the alignment to the model using an HMM
(the \prog{sub yes} part) and then align the region of the model that
spans from start to end \emph{globally} to the sequence (the
\prog{global} part). In this case, \emph{global} alignment means that
the program is forced to align the full model region from start to end
to the sequence (it is \emph{not} allowed to skip large parts of the
model without large score penalties as it would if \emph{local}
alignment was being performed). The \prog{bands} column tells us that
bands (constraints) from an HMM alignment will be used to accelerate
alignment to the CM. This is explained more in Chapter 8 of
\cite{Nawrocki09b}. The \prog{tau} column reports the probability loss
allowed when computing the HMM bands. In this case, \prog{1e-07}
probability mass is allowed outside each band.

The next section includes per-sequence information on the alignment
that was created:

\begin{sreoutputwide}
#                                                                  bit scores                            
#                                                               ------------------                       
#   seq idx  seq name                                      len     total    struct  avg prob      elapsed
# ---------  ------------------------------------------  -----  --------  --------  --------  -----------
          1  00052::Halobacterium_sp.::AE005128           1473   2335.05    290.70     1.000  00:00:01.00
          2  00013::Methanobacterium_formicicum::M36508   1476   2331.84    262.88     0.999  00:00:01.01
          3  00004::Nanoarchaeum_equitans::AJ318041        865   1294.71    170.40     0.999  00:00:00.46
          4  00121::Thermococcus_celer::M21529            1486   2430.39    249.92     0.998  00:00:01.02
          5  00115::Pyrococcus_furiosus::U20163|g643670    922   1496.81    138.96     0.997  00:00:00.51
\end{sreoutputwide}

We'll go through each of these columns:

\begin{wideitem}
\item[\emprog{seq idx}] the index of the sequence in the file.

\item[\emprog{seq name}] the name of the sequence.

\item[\emprog{len}] length of the sequence; the full sequence hit from 
  the search stage is aligned, no trimming of ends is permitted, as it was in the search
  stage with \prog{cmsearch}.

\item[\emprog{total}] the bit score of the CM alignment. For more
  information, see section 5 of the \sft{infernal} User's Guide \cite{Nawrocki09}.

\item[\emprog{struct}] the secondary structure score component of the
  \prog{total} bit score. These are the added bits that are due solely
  to the modeling of the consensus secondary structure of the
  molecule by the CM\@. 
  
\item[\emprog{avg prob}] the average posterior labeling, or confidence
  estimate, of the aligned nucleotides. The higher this value is the less
  ambiguous and more well-defined the alignment is. The highest this
  can possibly be is \prog{1.000}, which means very nearly 100\% of
  the probability mass of the alignment to the model is contained in
  the single, optimally accurate alignment that was reported by the
  program. In other words, the reported alignment receives a
  significantly higher score than any other alternative alignment. The
  program derives this value by evaluating the score of every possible
  alignment (consistent with the HMM bands) of the sequence to the
  model, and comparing the best, optimal score versus all of the
  rest. 
  %This is described in a bit more detail in section~\ref{section:chap9}.

\item[\emprog{elapsed}] the amount of actual time (wall time) it took
  the program to align this sequence. In general, less well-defined
  alignments with lower \prog{avg prob} will take longer than more
  well-defined ones. This is because the HMM bands are usually tighter
  and act as stricter constraints to the CM alignment when the
  alignment is well-defined. Tighter bands lead to quicker alignments
  because fewer possible alignments to the CM must be considered.
\end{wideitem}

\subsubsection{.stk suffixed files}
The \prog{.stk} suffixed files are Stockholm-formatted alignment
files. These are the alignments generated by \prog{cmalign}. The
statistics in the \prog{.cmalign} suffixed files correspond to these
alignments. One alignment is created for each model that was the
best-matching model to at least one sequence in \prog{ssu-align}'s
search stage. An explanation of Stockholm alignments can be found at
the beginning of the Tutorial, in section~\ref{sec:tutorial}.

\subsubsection{.scores suffixed files}

The \prog{.scores} file are meant to be a useful summary file.
They contain various statistics from each of the other output files
for every sequence in the original input sequence file. 

Take a look at the file \prog{myseqs.scores}:

\begin{sreoutputtinywide}
#                                                                        best-matching model                  second-best model 
#                                                         -------------------------------------------------  -------------------
#     idx  sequence name                                  model name   beg   end    CM sc   struct   HMM sc  model name   HMM sc  HMMdiff
# -------  ---------------------------------------------  ----------  ----  ----  -------  -------  -------  ----------  -------  -------
        1  00052::Halobacterium_sp.::AE005128             archaea        1  1473  2335.05   290.70  2080.08  bacteria     471.45  1608.63
        2  00013::Methanobacterium_formicicum::M36508     archaea        1  1476  2331.84   262.88  2108.16  bacteria     693.10  1415.06
        3  00004::Nanoarchaeum_equitans::AJ318041         archaea        1   865  1294.71   170.40  1112.07  bacteria     255.78   856.29
        4  00121::Thermococcus_celer::M21529              archaea      202  1687  2430.39   249.92  2223.71  bacteria     632.38  1591.33
        5  00115::Pyrococcus_furiosus::U20163|g643670     archaea      922     1  1496.81   138.96  1354.25  bacteria     378.49   975.76
        6  00035::Bacteroides_fragilis::M61006|g143965    bacteria       5  1537  2136.76   382.95  1797.77  archaea      576.22  1221.55
        7  01106::Bacillus_subtilis::K00637               bacteria       1  1552  2465.43   289.74  2212.50  archaea      768.61  1443.89
        8  00072::Chlamydia_trachomatis.::AE001345        bacteria       1   879  1215.64   183.53  1030.27  archaea      212.41   817.86
        9  01351::Mycoplasma_gallisepticum::M22441        bacteria     881     5  1165.86   217.01   940.15  archaea       95.30   844.85
       10  00224::Rickettsia_prowazekii.::AJ235272        bacteria      93  1594  2278.46   281.43  2037.49  archaea      613.67  1423.82
       11  01223::Audouinella_hermannii.::AF026040        eukarya        1  1770  2682.07   221.33  2481.69  archaea      222.16  2259.53
       12  01240::Batrachospermum_gelatinosum.::AF026045  eukarya        1  1764  2665.16   217.20  2463.49  archaea      219.14  2244.35
       13  00220::Euplotes_aediculatus.::M14590           eukarya        1  1082  1213.07    78.14  1103.29  archaea       96.64  1006.65
       14  00229::Oxytricha_granulifera.::AF164122        eukarya      600     1   851.30    23.53   788.12  archaea       86.31   701.81
       15  01710::Oryza_sativa.::X00755                   eukarya       75  1886  2687.06   142.17  2570.56  archaea      202.90  2367.66
\end{sreoutputtinywide}

There are four rows containing column headings prefixed with
\prog{\#}. Then there are 15 data rows, one for each sequence in the
input sequence file \prog{seed-15.fa}. Data rows are
separated into 12 columns:

\begin{wideitem}
\item[\emprog{idx}] the index of the sequence in the file.

\item[\emprog{sequence name}] the name of the sequence.
\end{wideitem}

The next 6 columns all describe the \emph{best-matching} model for the
sequence. This is the model that assigned the highest primary
sequence-based local profile HMM alignment score to the sequence.  If
no model aligned the sequence with a score higher than the minimum
threshold of 50 bits then the sequence was skipped and not aligned,
and all these columns will read \prog{-}. There are no such sequences
in the \prog{seed-15.fa} file, but if there were, an additional
\prog{.nomatch} file would have been created, as described
below. (Note: the minimum bit score
threshold value can be changed to \prog{<x>} using the
\prog{ssu-align} command-line option \prog{-b <x>}, as explained in
the \prog{ssu-align} anual page at the end of this guide).

\begin{wideitem}
\item[\emprog{model name}] name of best-matching model.

\item[\emprog{beg}] first nucleotide in the maximal
  scoring local HMM alignment to the best-matching model.

\item[\emprog{end}] final nucleotide in the maximal
  scoring local HMM alignment to the best-matching model.

\item[\emprog{CM sc}] the CM bit score for the best-matching model
  assigned to the sequence from positions \prog{beg} to \prog{end}.

\item[\emprog{struct}] the number of extra bits included in the
  CM bit score that are dervied from the secondary structure component
  of the model.

\item[\emprog{HMM sc}] the HMM bit score for the local alignment of
  the best-matching model to the sequence from positions \prog{beg} to
  \prog{end}.
\end{wideitem}

For more information on bit scores, see section 5 of the \sft{infernal}
User's Guide \cite{Nawrocki09}. 

The next 2 columns describe the \prog{second-best model}. This is the
model that assigned the second-highest primary sequence-based local 
profile HMM alignment score to the sequence. If only one models score
exceeded the minimum of 50 bits then these columns will each read
``\prog{-}''.

\begin{wideitem}
\item[\emprog{model name}] name of second-best-matching model.

\item[\emprog{HMM sc}] the HMM bit score for the local alignment of
  the second-best-matching model to the sequence. This alignment was
  not necessarily from \prog{beg} to \prog{end} (those were the
  coordinates of the alignment to the best-matching model). The
  sequence coordinates of the second-best model's alignment can be
  found in the file \prog{myseqs.tab}.
\end{wideitem}

The final column, \prog{HMMdiff}, reports the score difference between
the best-matching model HMM alignment and the second-best matching
model HMM alignment. This is included because it is an indication of
how clearly ``homologous'' the sequence is to the best-matching model
rather than to the second-best-matching model. The higher this score
difference is the more obvious it is that the sequence falls within
the sequence diversity represented by the best-matching model.
Sequences that are phylogenetically novel and do not obviously match
any single model much better than any other one should have relatively
small score differences in this column.

\subsubsection{.nomatch suffixed files}

A \prog{.nomatch} file simply lists the sequences in the input
sequence file that do not match to any model, one sequence name per
line. For a sequence to match to a model it must score above the
minimum profile HMM bit score threshold to at least one model. By
default, this threshold is 50 bits, but it can be changed to
\prog{<x>} bits with the \prog{-b <x>} option to \prog{ssu-align}. All
15 sequences in the tutorial file \prog{seed-15.fa} score above this
threshold to at least one model, so no \prog{.nomatch} file is
created. 

\subsection{.sum suffixed files}

The \prog{.sum} files include the text reported to the screen
(standard output) by \prog{ssu-align}. These files serve as a
reference to remind the user how \prog{ssu-align} was run
(parameters, input file names, etc.). All six programs in the
\textsc{ssu-align} package generate specifically named \prog{.sum}
suffixed files. For example, \prog{ssu-mask} generates a file ending
with the suffix \prog{.ssu-mask.sum}.

As a special case, when \prog{ssu-prep} is used to create parallel
jobs of \prog{ssu-align}, a \prog{ssu-prep.sum} file is created. Then,
when the \prog{ssu-prep}-generated shell script that executes the
parallel alignment jobs is run, a \prog{ssu-align.sum} file will be created.
This \prog{ssu-align.sum} file will contain the 
\prog{ssu-align.sum} files from all parallel alignment jobs
concatenated together, followed by a section describing the merge
performed automatically by the final job and alignment
statistics summarizing all parallel jobs.

\prog{ssu-align.sum} files include more information than the other
program's \prog{.sum} files. Specifically, they include two sections
labelled \prog{Summary statistics} and \prog{Speed statistics} which
summarizes the number of sequences in the input target sequence file
that match each model, and how fast the program completed the search
and alignment stages, respectively. Below is a short description of
each of the fields in the \prog{Summary statistics} section of the
\prog{myseqs.ssu-align.sum} file we just created:

\begin{sreoutput}
# Summary statistics:
#
# model or       number  fraction        average   average               
# category      of seqs  of total         length  coverage    nucleotides
# ------------  -------  --------  -------------  --------  -------------
  *input*            15    1.0000        1350.60    1.0000          20259
#
  archaea             5    0.3333        1244.40    0.9546           6222
  bacteria            5    0.3333        1268.60    0.9793           6343
  eukarya             5    0.3333        1405.60    0.9675           7028
#
  *all-models*       15    1.0000        1306.20    0.9671          19593
  *no-models*         0    0.0000              -         -              0
\end{sreoutput}

\begin{wideitem}
\item[\emprog{model or category}] the name of the model or category
  the row pertains to. Categories include \prog{*input*}, the set of 
  all full length target sequence file used as input; \prog{*all-models*},
  the set of sequences that match any model; and \prog{*no-models*}, the set of
  sequences that do not match any model above threshold. 

\item[\emprog{number of seqs}] number of seqs that belong in a
  category, or were best-matches to a model.

\item[\emprog{fraction of total}] the number of sequences in this
  model/category divided by the total number of sequences listed in the
  \prog{*input*} category.

\item[\emprog{average length}] the average sequence length of the
  model/category. For models, this is the average length of the
  subsequences that survive the search stage, which are not
  necessarily the full length sequences in the input sequence file.

\item[\emprog{average coverage}] length of suriviving sequences
  divided by the full length of those sequences in the input sequence
  file. 

\item[\emprog{nucleotides}] summed length of suriviving sequences in
  model/category. This is equal to the product of the values in
  \prog{number of seqs} and \prog{average length} columns.
\end{wideitem}

Now, take a look at the \prog{Speed statistics} section. Each field in
this table is described below:

\begin{sreoutput}
# Speed statistics:
#
# stage      num seqs  seq/sec  seq/sec/model    nucleotides    nt/sec
# ---------  --------  -------  -------------  -------------  --------
  search           15    1.159          0.386          20259    1565.5
  alignment        15    0.972          0.972          19593    1270.0
\end{sreoutput}

\begin{wideitem}
\item[\emprog{stage}] the stage of the program this row pertains to,
  either \prog{search} or \prog{alignment}.

\item[\emprog{num seqs}] the number of sequences processed by this stage.

\item[\emprog{seq/sec}] the number of sequences processed by this
  stage per second.

\item[\emprog{seq/sec/model}] the number of sequences processed by this
  stage per second per model. This is the value in the \prog{seq/sec}
  column divided by the number of models used to search for or align
  the sequences. Remember that in the \prog{search} stage, all 3
  models are used by default, whereas only 1 model is used in the
  \prog{alignment} stage (figure~\ref{fig:strategy}).

\item[\emprog{nucleotides}] summed length of all sequences processed
  by this stage.

\item[\emprog{nt/sec}] number of nucleotides processed by this stage
  per second.
\end{wideitem}

\subsection{.log suffixed files}

The \prog{.log} files include information on all of the system
commands run by \prog{ssu-align}. As with \prog{.sum} suffixed files,
program specific \prog{.log} suffixed files are created by each of the
six \textsc{ssu-align} programs. For example, \prog{ssu-mask}
generates a log file ending with the suffix \prog{.ssu-mask.log}.


As a special case, when \prog{ssu-prep} is used to create parallel
jobs of \prog{ssu-align}, a \prog{ssu-prep.log} file is created. Then,
when the \prog{ssu-prep}-generated shell script that executes the
parallel alignment jobs is run, a \prog{ssu-align.log} file will be created.
This \prog{ssu-align.log} file will contain the 
\prog{ssu-align.log} files from all parallel alignment jobs
concatenated together.

The \prog{.log} files list all of the commands that were executed by
the program along with their STDERR and sometimes their STDOUT output
(for example, the \prog{cmsearch} and \prog{cmalign} commands), as
well as the ``globals hash'', a set of variables defined in the
\textsc{ssu-align} PERL module \prog{ssu.pm} which is installed in the
\$SSUALIGNDIR. The \prog{.log} files should mainly be useful to expert 
users or developers who are trying to figure out why a specific run of a
program failed, or how to reproduce a specific step executed by the
program (e.g. the \prog{cmsearch} step of \prog{ssu-align}).

\subsection{ssu-mask output files}

The \prog{ssu-mask} example from the tutorial is executed with:

\user{ssu-mask myseqs} 

This command generates three output files per-domain, each of which is
briefly explained below. These files are also discussed in the
Tutorial section. 

\subsubsection{.mask suffixed files}
Mask files are simple one-line text files that correspond to a
specific domain model/alignment file and include only \prog{0} and
\prog{1} characters. Each character corresponds to a consensus column
of the model/alignment. For example, the
\prog{myseqs.archaea.mask} file contains 1508 characters, one per
consensus column of the default \textsc{ssu-align} archaeal model
described in section~\ref{sec:models}. A \prog{0} at position \emph{x}
indicates that consensus position \emph{x} is excluded from the mask
and will be removed from the alignment when the mask is applied. A
\prog{1} at position \emph{x} indicates that it is included by the
mask and will be kept when the mask is applied. A consensus position
of an alignment is one where the \prog{\#=GC RF} annotation is a
nongap (see section~\ref{sec:background} for more detail).

\subsubsection{.mask.pdf and .mask.ps suffixed files}
These are structure diagram files that display a mask on the consensus
secondary structure of a domain. Pink positions are excluded by the
mask. Black positions are included by the mask. Either \prog{.pdf} or
postscript \prog{.ps} files will be created, but not both. If you have
the program \prog{ps2pdf} in your path, \prog{.pdf} files will be
created, otherwise postscript will be. One difference between
these filetypes is their sizes, the PDF files created by
\prog{ssu-mask} are only about 10\% the size of the postscript files.

\subsubsection{.mask.stk suffixed files}
The \prog{.mask.stk} files are Stockholm-formatted alignment files
that include a subset of the columns of the alignments created by
\prog{ssu-align}. The consensus columns excluded by the mask have been
removed. Additionally, all insert columns have been removed. An insert
column is one which is a gap in the reference (\prog{\#=GC RF})
annotation of the alignment. For more information on inserts, see
section~\ref{sec:background}.

\subsubsection{.list suffixed files}
\prog{.list} suffixed files are only generated if \prog{ssu-mask} is
executed using the \prog{--list} option. A list file is generated for
a specific alignment file, and it simply lists the name of each
sequence in the alignment, one per line.

\subsubsection{.afa suffixed files}
\prog{.afa} suffixed files are FASTA-formatted alignment files that
are created by \prog{ssu-mask} only if the \prog{--stk2afa} option is
used. These files differ from unaligned FASTA files in that each
sequence may contain gaps, and all sequences are guaranteed to be
the same length. \prog{.afa} files created by \prog{ssu-mask}
contain two types of gap characters. A \prog{-} character
is a gap in a consensus column of an alignment. A \prog{.} character
is a gap in an insert column of an alignment. See
section~\ref{sec:background} for more information on the distinction
between consensus and insert columns. Note that \prog{.afa} alignment
files do not include structure information that is included in
Stockholm alignment files.

\subsection{ssu-draw output files}

The \prog{ssu-draw} example from the tutorial is executed with:

\user{ssu-draw myseqs} 

This command generates two output files per-domain, each of which is 
briefly explained below. 

\subsubsection{.pdf or .ps suffixed files}
These are structure diagram files that display either alignment
summary statistics (such as information content or frequency of gaps)
or individual aligned sequences on the consensus secondary structure
for a domain. They may be multiple pages. A brief description is
included at the top of each page. Along with each of these files, a
corresponding \prog{.drawtab} file is generated, as described below.
Either \prog{.pdf} or postscript \prog{.ps} files will
be created, but not both. If you have the program \prog{ps2pdf} in
your path, \prog{.pdf} files will be created, otherwise postscript
will be. One difference between
these filetypes is their sizes, the PDF files created by
\prog{ssu-draw} are only about 10\% the size of the postscript files.

\subsubsection{.drawtab suffixed files}
\prog{.drawtab} suffixed files are tab-delimited text files meant to
be easily parseable that contain the information displayed in a
corresponding \prog{.pdf} or \prog{.ps} structure diagram file. 
In these files, lines beginning with \prog{\#} are comment
lines. Comments at the beginning of \prog{.drawtab} files explain the
meaning of each of the tab-delimited fields in the non-comment lines. 

