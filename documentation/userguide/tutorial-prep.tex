\subsection{Splitting up large datasets to run in parallel on a cluster}

If you have access to a compute cluster you can partition your input
dataset and run \textsc{ssu-align} in parallel on multiple machines. A
motivated user could certainly write their own scripts to do this, but
the \prog{ssu-align} program has options to facilitate this
parallelization so you don't have to.  It also will create a script
that will merge the separate alignments created for each partition
into a single master alignment.

In this section we'll walk through an example of how to do this for a
small dataset.  The sequence file \prog{tutorial/seed-30.fa} includes
30 randomly chosen sequences from the complete set of seed sequences
from the 5 default models of \textsc{ssu-align} (there are 205 total
seed sequences, as shown in the table of seed alignment statistics in
section~\ref{section:seeds}). To follow this example, create a new
directory and copy \prog{ssu-align-0.1/tutorial/seed-50.fa}, the file
\prog{ssu-align-0.1/tutorial/janelia-sa-0p1.params} and the default parameters file
\prog{ssu-align-0.1/sa-0p1.params} there.

There are two ways \textsc{ssu-align} can split up a job into multiple
jobs to run in parallel. You can either specify the number of jobs
(\prog{<x>}) to create, or the number of sequences (\prog{<y>}) you
want each job to handle. These are invoked using the command-line
options \prog{-c <x>} and \prog{-n <y>} respectively.  In this example
we'll use the \prog{-c} method.

One important modification should be made to the parameters file
before you use \textsc{ssu-align} to split up a large job to run in
parallel on a cluster. You'll want to create a new parameters file
(you can use the default \prog{sa-0p1.params} as a starting point) and
add two lines to it defining a prefix string and suffix string that
should appear before and after the \prog{ssu-align} call if you were
submitting it as a job for the cluster to your local job
scheduler. For example, look at the final 3 lines of the file
\prog{tutorial/janelia-sa-0p1.params}:

\begin{sreoutput}
\$cluster_prefix = "qsub -N ssualign -o ssualign.out -b y -cwd -V -j y '";
\$cluster_suffix = "'";
1;
\end{sreoutput}

The definitions of the variables \prog{\$cluster\_prefix} and
\prog{\$cluster\_suffix} tell \prog{ssu-align} what prefix and suffix
strings, respectively, to add to the \prog{ssu-align} runs it will
create for each partition. These specific strings correspond to the
format used by the SGE (Sun Grid Engine) \prog{qsub} program we use
for scheduling jobs on our cluster here at Janelia Farm in Virginia.
Importantly, you'll need to change these to the format required by
your own compute resources.

If you want the jobs to run in parallel on your local machine, you
would replace these three lines with:
\begin{sreoutput}
\$cluster_prefix = "";
\$cluster_suffix = " &";
1;
\end{sreoutput}

Because our example dataset here is only 30 sequences we really don't
need a cluster to run it, so we don't need to define the
\prog{\$cluster\_prefix} and \prog{\$cluster\_suffix} variables in our
parameters file. We'll use the default \prog{sa-0p1.params}.

Let's say we want to split up our 30 sequences into 5 separate files
of 6 sequences each and run each set of 6 independently. 
To execute the script:

\user{ssu-align -c 5 ssu3-0p1.cm seed-30.fa seed-30-c5 sa-0p1.params}

The program finishes in about 1 second. It will print information on
the files it has created:

\newpage

\begin{comment}
# ssu-align :: define and align SSU rRNA sequences
# SSU-ALIGN 0.1 (June 2009)
# Copyright (C) 2009 HHMI Janelia Farm Research Campus
# Freely distributed under the GNU General Public License (GPLv3)
# - - - - - - - - - - - - - - - - - - - - - - - - - - - - - - - - - - - -
# command: ssu-align -c 5 ssu3-0p1.cm seed-30.fa seed-30-c5 sa-0p1.params
# date:    Wed Aug 19 18:57:15 2009
#
\end{comment}
\begin{sreoutput}
# Prep mode: Splitting up ssu-align job into 5 smaller jobs.
#
# output file name     description                                                 
# -------------------  ------------------------------------------------------------
  seed-30.fa.1         partition 1 fasta sequence file
  seed-30.fa.2         partition 2 fasta sequence file
  seed-30.fa.3         partition 3 fasta sequence file
  seed-30.fa.4         partition 4 fasta sequence file
  seed-30.fa.5         partition 5 fasta sequence file
  seed-30-c5.sh        shell script that will run ssu-align 5 times
  seed-30-c5.merge.pl  perl script to merge alignments when seed-30-c5.sh completes
  seed-30-c5.log       log file (*this* text printed to stdout)
#
# All output files created in directory ./seed-30-c5/
\end{sreoutput}

The first five fasta files are partitions of the original sequence
file \prog{seed-30.fa}. Each has 6 sequences in it. The file
\prog{seed-30-c5.sh} is a shell script that will execute
\prog{ssu-align} five times, once for each of the partitions. 
If we had defined \prog{\$cluster\_prefix} and
\prog{\$cluster\_suffix} this script would include those strings to
allow submission to a cluster. Since we did not our script will simply
execute \prog{ssu-align} three times in succession. The file
\prog{seed-30-c5.merge.pl} is a perl script we can use to merge the
individual alignments created by each of the five jobs together.
The \prog{seed-30-c5.log} file contains the exact text that was just
printed to the screen. All of these files were created in a new
directory called \prog{seed-30-c5}.

The next step is to run the \prog{seed-30-c5.sh} script, which will
execute \prog{ssu-align} three times. First, enter the
\prog{seed-30-c5/} directory, then type:

\prog{sh seed-30-c5.sh}

The output of the five \prog{ssu-align} runs will begin to print to
the screen. When one run finishes, the next will immediately
start. Each run should take about 20 seconds, so all five will take a
couple of minutes. 

Once they're done we can move to the next step: merging
the alignments. If we had defined \prog{\$cluster\_prefix} and
\prog{\$cluster\_suffix}, each line of our script would have submitted
a separate job to our cluster. The script would finish quickly, but
the jobs would still be running. \textbf{\emph{Important: }} you'll
need to wait until all the jobs are finished running on the
cluster before attempting to merge the alignments.

Continuing with our example, to merge the alignments:
\user{perl seed-30-c5.merge.pl}

This script will create a merged alignment for each model in the
original cm file (\prog{ssu3-0p1.cm} in this case) that was the
best-matching model for at least 1 sequence in any of the three runs.
For each such model, the script will merge two alignments at a time
using \textsc{infernal}'s \prog{cmalign} program (see the next section
on ``Merging multiple alignments together'' for an example)
until a single alignment with all the sequences is created. 

For example, in this case the alignment from jobs 1 and 2 would be
merged first, then the alignments from jobs 3 and 4 are merged. Now
three alignments exist, one from jobs 1+2, one from 3+4 and one from
5. Next 1+2 are merged with 3+4, and finally the resulting alignment
of 1+2+3+4 is merged with the alignment from job 5. If you look at the
code in \prog{seed-30-c5.merge.pl} it may look messy, but this
simple binary merging is all that it is doing. The intermediate
alignments are deleted once they are no longer needed as the script
proceeds. If you have a large number of alignments to merge and/or a
large number of total sequences this script may take a long time. 

In this case, 3 merged alignments are created, as reported by the
script:

\begin{sreoutput}
Merged alignment for archaea  CM saved to seed-30-c5.1-5.archaea.merged.stk
Merged alignment for bacteria CM saved to seed-30-c5.1-5.bacteria.merged.stk
Merged alignment for eukarya  CM saved to seed-30-c5.1-5.eukarya.merged.stk
\end{sreoutput}

These alignments will be 100\% identical to the alignments that would
have been created if we had not split up this job, but rather aligned
all 30 sequences with a single run of \prog{ssu-align}.

