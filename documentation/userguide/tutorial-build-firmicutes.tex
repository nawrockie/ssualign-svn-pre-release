\subsection{Using ssu-build to create a Firmicutes-specific SSU model}
\label{sec:tutorial-build-firmicutes}

The default archaeal, bacterial and eukaryotic models provided with
\sft{ssu-align} are meant to be general models that represent the SSU
sequence diversity in their respective domains. You might want to
create more more specific models that cover a tighter or broader
phylogenetic range, for example a particular bacterial phyla of
interest, or a single model that covers all SSU sequences. The
\prog{ssu-build} program allows users to create such models. 

\prog{ssu-build} takes as input a multiple sequence alignment, called
a \emph{seed} alignment, of SSU sequences and creates a CM file that
includes a model that represents the diversity from the seed
alignment. This CM file can be used individually to align sequences,
or combined, through simple file concatenation, with other CM files
and used to classify and create clade-specific SSU alignments.

In this section, we'll create two bacterial SSU models, a
\emph{Firmicutes} specific model, and a model that represents all
other non-\emph{Firmicutes} bacteria. We'll then combine these two
models with the default archaeal and eukaryotic models and use them to
classify and align sequences from a sample dataset. 

\subsubsection{Partitioning the default bacterial seed alignment}

For this demonstration, we'll create our \emph{Firmicutes} and
non-\emph{Firmicutes} bacterial model using the default bacterial seed
alignment. The first step is to identify which of the sequences in the
seed are \emph{Firmicutes}. To do this, I ran the FASTA file with the
unaligned bacterial seed sequences in \prog{seeds/bacteria-0p1.fa}
through the \sft{RDP-classifier} tool using default parameters
\cite{Wang07}
on the \sft{RDP} website:
\htmladdnormallink{http://rdp.cme.msu.edu/classifier}{http://rdp.cme.msu.edu/classifier}\footnote{Version
  2.2 of \sft{RDP-classifier} used with \sft{RDP} training set 6.}
Seven of the 93 sequences are classified as \emph{Firmicutes} with an
80\% bootstrap threshold. These seven sequences are:
\begin{sreoutput}
01088::Bacillus_halodurans.::AB013373
01106::Bacillus_subtilis::K00637
01382::Clostridium_perfringens::M69264
01375::Clostridium_tetani::X74770|g509282
01295::Lactococcus_lactis_subsp._lactis.::AE006456
01252::Staphylococcus_aureus::L36472|g567883__bases_8528_to_10082_
01305::Streptococcus_pyogenes::AE006473
\end{sreoutput}

We'll use these sequences to build a \emph{Firmicutes} specific model. 
For demonstration purposes, we'll omit the \emph{Bacillus subtilis}
sequence for reasons that will become clear shortly.
%The other six are listed in the file \prog{firm-6.txt} in the
%\prog{ssu-align-0.1/tutorial/} directory. All seven are listed in the
%file \prog{firm-7.txt}. 
The remaining 86 seed sequences are non-\emph{Firmicutes}, we'll build 
a separate model from these. 

To build the models, we first need to extract the relevant aligned sequences
from the bacterial seed alignments using \prog{ssu-mask}. We'll use
the list files \prog{firm-7.list} and \prog{firm-6.list} in
\prog{ssu-align-0.1/tutorial} to do this. The \prog{firm-7.list} file
includes the names of the seven \emph{Firmicutes} sequences, one per
line, and \prog{firm-6.list} includes the same names but without the
\emph{Bacillus subtilis} sequence. 

To create the six sequence \emph{Firmicutes}-only alignment, go to the
\prog{tutorial/} directory and execute:

\user{ssu-mask -a --seq-k firm-6.list --key-out firm-6 ../seeds/bacteria-0p1.stk}

The \prog{--seq-k} option tells \prog{ssu-mask} to keep only the
six sequences listed in \prog{firm-6.list}
The program will report that the new alignment file
\prog{bacteria-0p1.firm-6.seqk.stk} has been created. 

Now to create the 86 sequence non-\emph{Firmicutes} alignment do:

\user{ssu-mask -a --seq-r firm-7.list --key-out nonfirm-86 ../seeds/bacteria-0p1.stk}

The \prog{--seq-r} option tells \prog{ssu-mask} to remove only the
seven sequences listed in \prog{firm-7.list}
The program will report that the new alignment file
\prog{bacteria-0p1.nonfirm-86.seqk.stk} has been created. 

%Normally, at this stage an expert user would manually refine these
%alignments, or expand them by adding more sequences. 

Now, we can use \prog{ssu-build} to build models from the alignments
we've just created. 

To build the \emph{Firmicutes} model, do:

\user{ssu-build -n firmicutes -o firmicutes.cm bacteria-0p1.firm-6.seqk.stk}

This will create the file \prog{firmicutes.cm} which includes a CM
named \emph{firmicutes}. 

To build the non-\emph{Firmicutes} model, do:

\user{ssu-build -n bacteria-nonfirm -o bacteria-nonfirm.cm
  bacteria-0p1.nonfirm-86.seqr.stk}

This will create the file \prog{bacteria-nonfirm.cm} which includes a CM
named \emph{bacteria-nonfirm}. 

We can use both of these models simultaneously to differentiate
\emph{Firmicutes} SSU sequences from other bacterial SSU sequences by
concatenating them together: 

\user{cat firmicutes.cm bacteria-nonfirm.cm > my.cm}

Since we'd also like to be able to differentiate archaeal and
eukaryotic sequences, we can also append those default models:

\user{cat ../seeds/archaea-0p1.cm ../seeds/eukarya-0p1.cm >> my.cm}

Now the file \prog{my.cm} includes four CMs. Let's run
\prog{ssu-align} using this model on the \prog{seed-15.fa} dataset
we've been using throughout this tutorial:

\user{ssu-align -f -m my.cm seed-15.fa myseqs3}

\begin{sreoutput}
# Stage 1: Determining SSU start/end positions and best-matching models...
#
# output file name                  description                                        
# --------------------------------  ---------------------------------------------------
  myseqs3.tab                       locations/scores of hits defined by HMM(s)
  myseqs3.firmicutes.hitlist        list of sequences to align with firmicutes CM
  myseqs3.firmicutes.fa                   1 sequence  to align with firmicutes CM
  myseqs3.bacteria-nonfirm.hitlist  list of sequences to align with bacteria-nonfirm CM
  myseqs3.bacteria-nonfirm.fa             4 sequences to align with bacteria-nonfirm CM
  myseqs3.archaea.hitlist           list of sequences to align with archaea CM
  myseqs3.archaea.fa                      5 sequences to align with archaea CM
  myseqs3.eukarya.hitlist           list of sequences to align with eukarya CM
  myseqs3.eukarya.fa                      5 sequences to align with eukarya CM
\end{sreoutput}

Earlier in this tutorial, we used the default models to classify and
align these sequences and 5 sequences were assigned to each of the
three domains. In this search, 1 of the 5 bacterial sequences has been
classifed as a \emph{Firmicutes} sequence. The
\prog{myseqs3.firmicutes.hitlist} file contains the name of the
sequence:

\begin{sreoutput}
# target name                        start    stop     score
# --------------------------------  ------  ------  --------
  01106::Bacillus_subtilis::K00637       1    1552   2304.65
\end{sreoutput}

This is the \emph{Bacillus subtilis} sequence from the bacterial seed
that we purposefully omitted from the \emph{Firmicutes}
seed. \prog{ssu-align} has correctly identified it as a
\emph{Firmicutes} sequence. If we had included this sequence in the
seed, its correct identification would have easier for the program, so
it was omitted to make the task more challenging.

In this section we've crudely partitioned the bacterial seed alignment
into two smaller seed alignments and built a separate model from each,
without modifying either seed. A better, but more labor intensive,
strategy would be to manually examine and modify the seeds before constructing a
CM from them. Given the limited context of the sequences in the seed,
it may be clear that certain alignment regions should be refined,
basepairs should be added or removed. Also, more sequences
can be added to make the model more robust, while six is probably too
few, more than 100 is typically not necessary.

