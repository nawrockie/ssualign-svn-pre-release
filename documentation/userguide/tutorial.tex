\section{Basic tutorial: defining and aligning SSU sequences using \textsc{SSU-align}}

Here is a tutorial walk-through of a small project with
\software{SSUalign}. This tutorial shows how to use the program for
it's most basic and fundamental purpose, creating multiple
alignments of SSU rRNA sequences. 

\subsection{Files used in this tutorial}

The subdirectory \prog{/tutorial-basic} in the \software{SSU-align}
contains the files used in this tutorial, they are:

  \begin{sreitems}{}
  \item[\prog{ssu.default.0p1.cm}] A covariance model (CM) file that
    defines five SSU rRNA CMs: an archael model, a bacterial model, a
    choloroplast model, a eukaryotic model and a metazoan
    mitrochondria model. These are the five default models used by
    \textsc{SSU-align}. These models are explained in section 4.
  \item[\prog{rocks.fa}] SSU rRNA sequences from an environmental
    survey sequencing project of microbes living in the pore space of
    rocks in the Rocky Mountains by J.J. Walker and Norm Pace
    \cite{Walker07}. 
  \item[\prog{1p0.params}] A file containing paths to
    \textsc{infernal} executable files that \textsc{SSU-align} needs
    to run. You will likely need to change these paths to point to
    where you've installed the \prog{cmsearch} and \prog{cmalign}
    programs (these are created in 'infernal-1.0/src/' after building
    \textsc{infernal} version 1.0 with 'sh ./configure; make;') and
    the \prog{esl-sfetch} program (which is created in
    'infernal-1.0/easel/miniapps/' after building \textsc{infernal}
    version 1.0).
  \end{sreitems}

Create a new directory that you can work in, and copy all the files in
\prog{tutorial-basic} there. I'll assume for the following examples that
you've installed the \software{SSU-align} script in your path; if not,
you'll need to give a complete path name to the script
(e.g. something like
\newline
\prog{/usr/people/nawrocki/ssualign/src/ssu-align} 
instead of just \prog{ssu-align}).

\subsection{An example run of \textsc{ssu-align}}

The file \prog{rocks.fa} contains 588 SSU sequences
\cite{Walker07}. \textsc{SSU-align} is designed to create structural
alignments of SSU sequences from studies like this one. To run it,
execute the following command:

\user{ssu-align ssu.default.0p1.cm rocks.fa myrun 1p0.params}\\

The program will print a header describing the program version used,
command used, current date, and some other information. Then it will
begin stage 1:

\begin{sreoutput}
# ssu-align :: define and align SSU rRNA sequences
# SSU-ALIGN 0.1 (June 2009)
# Copyright (C) 2009 HHMI Janelia Farm Research Campus
# Freely distributed under the GNU General Public License (GPLv3)
# - - - - - - - - - - - - - - - - - - - - - - - - - - - - - - - - - - - -
# command: /groups/eddy/home/nawrockie/ssualign/ssu-align ssu.default.1p0.cm rocks.fa myrun 1p0.params
# date:    Wed Jun 17 05:37:27 2009
#
# Stage 1: Determining SSU start/end positions and best matching models.
\end{sreoutput}

In stage 1, the program scans the input sequences with each of the
five models in the CM file \prog{ssu.default.0p1.cm}. This has two
purposes.  First, it classifies each sequence by determining which
model gives each sequence the highest primary sequence-based alignment
score using a profile HMM. Secondly, it defines the start and end
points of the SSU sequences.

Stage 1 takes about 5 minutes on this dataset. When it finishes you'll
see: 

\begin{sreoutput}
# Stage 1: Determining SSU start/end positions and best matching models.
#
# output file name               description                                   
# -----------------------------  ----------------------------------------------
  myrun.tab                      locations/scores of hits defined by HMM(s)
  myrun.Archaea.hits.list        list of sequences to align with Archaea CM
  myrun.Archaea.hits.fa               48 sequences to align with Archaea CM
  myrun.Bacteria.hits.list       list of sequences to align with Bacteria CM
  myrun.Bacteria.hits.fa             341 sequences to align with Bacteria CM
  myrun.Chloroplast.hits.list    list of sequences to align with Chloroplast CM
  myrun.Chloroplast.hits.fa          199 sequences to align with Chloroplast CM
\end{sreoutput}

This lists and briefly describes the 7 new files the script created in
a newly created subdirectory of the current working dir called
\prog{myrun/}. The first file \prog{myrun.tab} is output from
\textsc{infernal}'s \prog{cmsearch} program. The other 6 files are
model-specific, two for each model that was the best-matching model to
at least 1 sequence in the input target sequence file
\prog{rocks.fa}. The \prog{.hits.list} suffixed files contain a list
of the sequences that match best to the model, and the \prog{.hits.fa}
suffixed files are those actual sequences. There were no sequences
that best-matched the eukaryotic model or the metazoan mitochondria
model, so no model-specific files were created for those two models.
Each of these file types is explained in more detail below in the
``Description of output files'' section, but for now we'll continue
following the output of our example \prog{ssu-align} run.

The program will now proceed to stage 2, the alignment stage. This
stage serially progresses through each model that was the
best-matching model for at least 1 sequence and uses the model to
align best-matching sequences. The alignments are computed by scoring
a combination of both sequence and secondary structure conservation,
as opposed to the scoring in stage 1 which only used sequence
conservation. When the alignment to each model finishes, a list of two
newly created files will appear on the screen. For this example,
alignment to all three models takes about 6 minutes. When it finishes
you'll see:

\begin{sreoutput}
#
# Stage 2: Aligning each sequence to it's best matching model.
#
# output file name               description
# -----------------------------  ----------------------------------------------
  myrun.Archaea.cmalign.stk      Archaea alignment
  myrun.Archaea.cmalign          Archaea cmalign output
  myrun.Bacteria.cmalign.stk     Bacteria alignment
  myrun.Bacteria.cmalign         Bacteria cmalign output
  myrun.Chloroplast.cmalign.stk  Chloroplast alignment
  myrun.Chloroplast.cmalign      Chloroplast cmalign output
  myrun.scores                   list of CM/HMM scores for each sequence
  myrun.log                      log file (*this* text printed to stdout)
#
# All output files created in directory ./myrun/
#
# CPU time (search):     00:04:35
# CPU time (alignment):  00:06:19
# CPU time (total):      00:10:55
\end{sreoutput}

The actual alignments are the \prog{.cmalign.stk} suffixed
files. These were created by the \textsc{infernal} program
\prog{cmalign}. The \prog{cmalign} output is in the \prog{.cmalign}
suffixed files.  As in stage 1, these files were created in
the \prog{./myrun/} subdirectory. 

\subsection{Description of output files}

Now we'll go through each of the output file types
created by\prog{ssu-align}, starting with the alignments.

Take a look at the archaeal alignment we just created in
\prog{myrun/myrun.Archaea.cmalign.stk}. 

This alignment includes consensus secondary structure annotation and
is in \emph{Stockholm format}. 
Stockholm format, the native alignment format used by \software{hmmer} and
\software{Infernal} and the \database{Pfam} and \database{Rfam}
databases, is documented in detail in the \software{Infernal} User's
Guide which is included in this distribution in
\prog{infernal/documentation/userguide.pdf}.

For now, what you need to know about the key features of the alignment file is:
\begin{itemize}

\item The alignment is in an interleaved format, like other
  common alignment file formats such as \software{clustalw}.
  Lines consist of a name, followed by an aligned sequence;
  the alignment is split into blocks separated by blank lines.

\item Gaps are indicated by the characters ., \_, -, or \verb+~+.
  qNotice that the first few blocks of the alignment are 100% gaps.
  This is because the sequences in \prog{rocks.fa} are not full length
  SSU sequences, but rather partial sequences obtained using PCR
  primers that target well conserved regions within the SSU
  molecule. In this alignment you'll have to scroll down to about line
  1300 before you see an aligned residue.

\item Special lines starting with {\small\verb+#=GR+} followed by a
  sequence name and then {\small\verb+POST+} contain posterior
  probabilities for each aligned residue for the sequence they
  correspond to. These are confidence estimates in the correctness of
  the alignment.  The POSTX. row indicates the ’tens’ place of the
  confidence estimate while POST.X row indicates the ’ones’ place. So
  the confidence estimate for a residue with 9 in the POSTX. row, and
  7 in the POST.X row to two significant digits 97%. This means that
  if you sampled alignments from the posterior distribution of all
  possible alignments of this sequence to the model, about 97% of the
  time that residue would appear in that of the alignment. One special
  case: if the posterior probability is very nearly 100% (it’s
  difficult to be more precise on the exact percentage due to
  numerical precision issues) the annotated posterior values will be
  ``*'' characters in both the tens and one places. These confidence
  estimates can be used to mask the alignment to remove columns with
  significant fractions of ambiguously aligned residues as demonstrated
  in the next section.

\item A special line starting with {\small\verb+#=GC SS_cons+}
  indicates the secondary structure consensus. Gap characters annotate
  unpaired (single-stranded) columns. Base pairs are indicated by any
  of the following pairs: \verb+<>+, \verb+()+, \verb+[]+, or
  \verb+[]+.

\item A special \"RF\" line starting with {\small\verb+#=GC RF+}
  indicates the consensus, or ReFerence, model. Gaps in the RF line
  are \emph{insert} columns, where at least 1 sequence has at least 1
  inserted residue between two consensus positions. Uppercase residues
  in the RF line are well conserved positions in the model; lowercase
  residues are less well conserved.
\end{itemize}

\subsection{Creating secondary structure diagrams that display alignment statistics}

SSU rRNA alignments are large and difficult to view in a meaningful
way. \textsc{ssu-align} includes a program \prog{esl-ssudraw} for
creating secondary structure diagrams that display statistics of a
particular alignment on the consensus secondary structure used to
align the model. To use \prog{esl-ssudraw} requires a template
postscript file of the consensus secondary structure. The template
files for the 5 default \textsc{ssu-align} version 0.1 seed models are
included, but constructing these required a significant amount of
work, and creating your own templates for different models would be
difficult. 

Let's create a diagram that shows the information content of each
position of our newly created archaeal alignment:

\prog{esl-ssudraw 




To convert the alignment to fasta format that includes gaps, you can use the
\prog{scripts/stk2aln\_fa.pl} script. 

\subsection{Pruning the alignment based on probabilistic confidence
  estimates}

