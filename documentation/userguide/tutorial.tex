\section{Tutorial}
\label{section:tutorial}

Here is a tutorial walk-through of some small projects with
\sft{ssu-align}. To follow along with this tutorial, move into the
\prog{tutorial/} subdirectory of the {ssu-align-0.1/} directory where
you unpacked and built the package.
The instructions in this tutorial assume that you have already
installed the package (see the Installation section)
and all of the \sft{ssu-align} executable programs
in your \$PATH. For example, you should be able to run
\prog{ssu-align} by simply typing \prog{ssu-align}.

\begin{comment}
\subsection{Files used in this tutorial}

In the first section of this tutorial we'll use the following files in
the \prog{tutorial} directory:

  \begin{sreitems}{}
  \item[\prog{seed-15.fa}] a sequence file containing
    fifteen SSU rRNA sequences, created specifically for use in this
    tutorial. These are full or partial sequences from the archaeal,
    bacterial and eukaryotic default \sft{crw} seed alignments used to
    build the default \software{ssu-align} models.
  \end{sreitems}
\end{comment}

\subsection{Creating, masking and visualizing SSU alignments}

We'll use a small dataset to demonstrate how the package works.
The file \prog{seed-15.fa} contains five archaeal sequences, five
bacterial sequences and five eukaryotic sequences from the
\sft{ssu-align} v0.1 seed alignments. These seed alignments
were derived from alignments from the \db{CRW} database
\cite{CannoneGutell02} as described in section~\ref{section:chap9}.

Pretend that \prog{seed-15.fa} is a set of SSU sequences obtained from
an environmental sampling survey and that we want to analyze. First,
we can run the \prog{ssu-align} program to classify each sequence to
its domain of life, and create an alignment for each domain:

\user{ssu-align seed-15.fa myseqs}\\

As you can see, the \sft{ssu-align} script takes two command line
arguments. The first is the target sequence file. The second is the
name of a directory that \sft{ssu-align} will create and place its
output files into. This directory should not yet exist. 

The program will first print a header describing the program version
used, command used, current date, and some other information. 
The following information printed to the screen:

\begin{sreoutput}
# ssu-align :: align SSU rRNA sequences
# SSU-ALIGN 0.1 (May 2010)
# Copyright (C) 2010 HHMI Janelia Farm Research Campus
# Freely distributed under the GNU General Public License (GPLv3)
# - - - - - - - - - - - - - - - - - - - - - - - - - - - - - - - - - - - -
# command: ssu-align seed-15.fa myseqs
# date:    Wed May  5 13:17:20 2010
#
# Validating input sequence file ... done.
#
# Stage 1: Determining SSU start/end positions and best-matching models...
\end{sreoutput}

In stage 1, the program scans the input sequences with each of the
three default SSU models. This has two purposes.  First, it classifies
each sequence by determining which model in the input CM file is its
``best-matching'' model, defined as
the model that gives the sequence the highest primary sequence-based
alignment score using a profile HMM. Secondly, it
defines the start and end points of the SSU sequences based on the
best-matching model's alignment.

Stage 1 takes about 20 seconds on this dataset (on an Intel Xeon 3.0
GHz processor, which I'll use for all example runs in this
guide). When it finishes you'll see: 

\begin{sreoutput}
# Stage 1: Determining SSU start/end positions and best-matching models...
#
# output file name         description                                
# -----------------------  -------------------------------------------
  myseqs.tab               locations/scores of hits defined by HMM(s)
  myseqs.archaea.hitlist   list of sequences to align with archaea CM
  myseqs.archaea.fa              5 sequences to align with archaea CM
  myseqs.bacteria.hitlist  list of sequences to align with bacteria CM
  myseqs.bacteria.fa             5 sequences to align with bacteria CM
  myseqs.eukarya.hitlist   list of sequences to align with eukarya CM
  myseqs.eukarya.fa              5 sequences to align with eukarya CM
\end{sreoutput}

This lists and briefly describes the seven new files the script created
in the newly created \prog{myseqs/} subdirectory of the current directory.
The content and format of these files are described
in detail in section~\ref{section:output}. For now a brief explanation
should be sufficient. The first file \prog{myseqs.tab} is output from
\sft{infernal}'s \prog{cmsearch} program. The other six files are
model-specific: two files for each model that was the best-matching
model for at least one sequence in the input target sequence file
\prog{seed-15.fa}. The \prog{.hitlist} suffixed files contain a list
of the sequences that match best to the model, and the \prog{.fa}
suffixed files are those actual sequences. If any of the models had
not been the best-matching model to at least one target sequence,
there would be no \prog{.hitlist} or \prog{.fa} files for that
model.

The program will now proceed to stage 2, the alignment stage. This
stage serially progresses through each model that was the
best-matching model for at least one sequence and uses the model to
align the best-matching sequences. The alignments are computed by scoring
a combination of both sequence and secondary structure conservation,
as opposed to the scoring in stage one which only used sequence
conservation. As the alignment to each model finishes, two new lines
of text, one for each of two newly created files, will appear on the
screen. For this example, alignment to all three models takes about 20
seconds. When it finishes you'll see:

\begin{sreoutput}
# Stage 2: Aligning each sequence to its best-matching model...
#
# output file name         description
# -----------------------  ---------------------------------------
  myseqs.archaea.stk       archaea alignment
  myseqs.archaea.cmalign   archaea cmalign output
  myseqs.archaea.ifile     archaea insert info
  myseqs.bacteria.stk      bacteria alignment
  myseqs.bacteria.cmalign  bacteria cmalign output
  myseqs.bacteria.ifile    bacteria insert info
  myseqs.eukarya.stk       eukarya alignment
  myseqs.eukarya.cmalign   eukarya cmalign output
  myseqs.eukarya.ifile     eukarya insert info
  myseqs.scores            list of CM/HMM scores for each sequence
\end{sreoutput}

The newly created alignments are the \prog{.stk} suffixed files. These
were created by \sft{infernal}'s \prog{cmalign} program. The
\prog{.cmalign} and \prog{.ifile} suffixed files were also output by
\prog{cmalign}. As in stage 1, these files were created in the
\prog{./myseqs/} subdirectory. 

\subsubsection{Description of alignments}

Section~\ref{section:output} contains more information
on \sft{ssu-align} output files, but for now we'll focus only on the new
alignments.  Take a look at the archaeal alignment we just created in
\prog{myseqs/myseqs.archaea.stk}.

This alignment includes consensus secondary structure annotation and
is in \emph{Stockholm format}. 
Stockholm format, the native alignment format used by \sft{hmmer} and
\sft{infernal} and the \db{pfam} and \db{rfam}
databases, is documented in detail in the \sft{infernal} user's
guide which is included as a PDF in \prog{infernal-1.01/}

For now, what you need to know about the key features of the alignment file is:
\begin{itemize}

\item The alignment is in an interleaved format, like other
  common alignment file formats such as \sft{clustalw}.
  Lines consist of a name, followed by an aligned sequence;
  the alignment is split into blocks separated by blank lines.

\item Gaps are indicated by the characters ., \_, -, or \verb+~+.
  Many SSU alignments will have large regions of 100\% gaps at the
  beginning and ends of the alignment. 
  This will happen if the sequences are 
  partial SSU sequences, such as those  obtained with PCR
  primers that target well conserved regions within the SSU
  molecule.

\item Special lines starting with {\small\verb+#=GR+} followed by a
  sequence name and then {\small\verb+PP+} contain posterior
  probabilities for each aligned nucleotide for the sequence they
  correspond to.  These are confidence estimates in the correctness of
  the alignment.  Figure~\ref{fig:ambiguity} in
  section~\ref{section:chap9} provides an example of confidence
  estimates and posterior probabilities.  Characters in PP rows have
  12 possible values: "0-9", "*", or ".". If ".", the position
  corresponds to a gap in the sequence. A value of "0" indicates a
  posterior probability of between 0.0 and 0.05, "1" indicates between
  0.05 and 0.15, "2" indicates between 0.15 and 0.25 and so on up to
  "9" which indicates between 0.85 and 0.95. A value of "*" indicates
  a posterior probability of between 0.95 and 1.0. Higher posterior
  probabilities correspond to greater confidence that the aligned
  nucleotide belongs where it appears in the alignment.  These
  confidence estimates can be used to mask the alignment to remove
  columns with significant fractions of ambiguously aligned nucleotides
  as demonstrated below.

\item A special line starting with {\small\verb+#=GC SS_cons+}
  indicates the secondary structure consensus. Gap characters annotate
  unpaired (single-stranded) columns. Base pairs are indicated by any
  of the following pairs: \verb+<>+, \verb+()+, \verb+[]+, or
  \verb+[]+.

\item A special ``RF'' line starting with {\small\verb+#=GC RF+}
  indicates the consensus, or ReFerence, model. Gaps in the RF line
  are \emph{insert} columns, where at least 1 sequence has at least 1
  inserted nucleotide between two consensus positions. Uppercase nucleotides
  in the RF line are well conserved positions in the model; lowercase
  nucleotides are less well conserved.
\end{itemize}

\subsection{Masking (removing columns from) alignments}

If your goal is to use a phylogenetic inference program to build trees
from alignments created by \sft{ssu-align}, you may want to mask
out columns of the alignment that may include misaligned nucleotides
first, and then only run the inference on the remaining columns where
you're confident the alignment is correct. 

\begin{comment}
SSU alignments are commonly used for phylogenetic inference to
characterize the diversity of microorganisms in an environmental
sample. Because alignment errors confound phylogenetic inference, 
it is first recommended to remove columns from, or mask, SSU
alignments to remove regions that are likely to contain at least some
errors. 
\end{comment}

The \prog{ssu-mask} program uses the posterior probabilities
(PP values) in the alignments to determine which alignment columns
contain a significant fraction of nucleotides that are aligned with
low confidence. It takes a single command-line argument, the name of
the directory created by \prog{ssu-align}. The directory must exist
within the current working directory. To run it for our example, do: 

\user{ssu-mask myseqs} 

As with \prog{ssu-align}, the program will print information to the
screen about what it is doing and the files it is creating:

\newpage 

\begin{sreoutput}
# Masking alignments based on posterior probabilities...
#
#                                                    mask    
#                                                ------------
# file name                 in/out  type  #cols  incl.  excl.
# ------------------------  ------  ----  -----  -----  -----
  myseqs.archaea.stk         input   aln   1511      -      -
  myseqs.archaea.mask       output  mask   1508   1449     59
  myseqs.archaea.mask.pdf   output   pdf   1508   1449     59
  myseqs.archaea.mask.stk   output   aln   1449      -      -
#
  myseqs.bacteria.stk        input   aln   1597      -      -
  myseqs.bacteria.mask      output  mask   1582   1499     83
  myseqs.bacteria.mask.pdf  output   pdf   1582   1499     83
  myseqs.bacteria.mask.stk  output   aln   1499      -      -
#
  myseqs.eukarya.stk         input   aln   2009      -      -
  myseqs.eukarya.mask       output  mask   1881   1634    247
  myseqs.eukarya.mask.pdf   output   pdf   1881   1634    247
  myseqs.eukarya.mask.stk   output   aln   1634      -      -
\end{sreoutput}

The 'file name' column includes file names, either input or output
files, as listed in the 'in/out' field, with type specified by the
'type' column: 'aln' for alignment, 'mask' for mask file, and 'pdf' or
'ps' for structure diagram file. The 'cols' column gives the number of
columns for the file. The 'incl.' and 'excl.' columns list the number
of consensus columns of the alignment that are included (not removed)
and excluded (removed), respectively, by the mask. For example, the
\prog{myseqs.archaea.stk} input alignment was initially 1511 total columns,
1508 of which were consensus columns; 59 of these 1508 were deemed
unreliable and removed by the mask based on the PP values in those
columns, and the remaining 1449 were not removed. The alignment file
\prog{myseqs.archaea.mask.stk} was created that includes only these 1449
columns, and a structure diagram showing the positions of the 1449
columns in the context of the full archaeal consensus structure was
written to \prog{myseqs.archaea.mask.pdf} \footnote{In your hands, a
postscript file \prog{myseqs.archaea.mask.ps} may be created instead
of this pdf file. This will happen if you do not have the
\prog{ps2pdf} program installed and in your PATH. See the manual
page for \prog{ssu-mask} for more information.}

After running this command, the three files with \prog{.mask.stk}
suffixes in the \prog{myseqs} directory are your masked
alignments. \textsc{ssu-align} has high confidence that the columns in
these alignments contain very few errors. 

\subsubsection{using precalculated masks for consistent masking of
  multiple datasets}

In the \prog{ssu-mask} example above, we calculated a mask
specifically based on our \prog{myseqs} alignments. Given a different
dataset of SSU sequences, the generated masks would very likely be
different (i.e. exclude different columns). This
per-alignment-specificity can be undesirable. For example, if you'd
like to directly compare alignments from multiple runs of
\textsc{ssu-align} you probably want to make sure all of the
alignments being compared contain an identical set of consensus
positions \footnote{This would also allow the alignments to be combined
easily; see the \prog{ssu-merge} manual page.}. \textsc{ssu-align}
contains a default, precalculated mask for each of the three
models. These masks were derived from very large alignments as
described in section X (TO DO). As we'll see in the next example, you
can use these masks by supplying the \prog{-d} option to
\prog{ssu-mask}. We'll also use the \prog{--key-out} option to add the
string 'default' to the names of the output files so we do not
overwrite the files we created in the previous example. This option
enables you to save multiple sets of alignments and masks in a single
directory:

\user{ssu-mask -d --key-out default myseqs}

\begin{sreoutput}
# Masking alignments using pre-existing masks...
#
#                                                    mask    
#                                                ------------
# file name                 in/out  type  #cols  incl.  excl.
# ------------------------  ------  ----  -----  -----  -----
  myseqs.archaea.stk         input   aln   1511      -      -
  archaea-0p1.mask           input  mask   1508   1376    132
  myseqs.archaea.mask.pdf   output   pdf   1508   1376    132
  myseqs.archaea.mask.stk   output   aln   1376      -      -
#
  myseqs.bacteria.stk        input   aln   1597      -      -
  bacteria-0p1.mask          input  mask   1582   1376    206
  myseqs.bacteria.mask.pdf  output   pdf   1582   1376    206
  myseqs.bacteria.mask.stk  output   aln   1376      -      -
#
  myseqs.eukarya.stk         input   aln   2009      -      -
  eukarya-0p1.mask           input  mask   1881   1343    538
  myseqs.eukarya.mask.pdf   output   pdf   1881   1343    538
  myseqs.eukarya.mask.stk   output   aln   1343      -      -
\end{sreoutput}

The output is similar to the previous example, but note that the
\prog{.mask} suffixed files were input rather than output (these files
are placed in your \$SSUALIGNDIR during installation), and the number
of columns included and excluded has changed.

There are several other options that can be supplied to
\prog{ssu-mask}, including \prog{-f} and \prog{-k} which gives you the
ability to use your own precalculated masks. See the \prog{ssu-mask}
manual page for more information. 

\subsubsection{converting Stockholm alignments to FASTA format}
With the ambiguously aligned regions of the alignment removed, you may
want to use the alignments as input to a phylogenetic inference
program, but not many of those programs can accept Stockholm formatted
alignments. You can convert the Stockholm alignments to aligned FASTA
using the \prog{ssu-mask} program by specifying the \prog{--stk2afa}
option on the command line:

\user{ssu-mask --stk2afa myseqs}

After running, three \prog{.afa} suffixed files will have been created
in the \prog{myseqs} directory.


\subsection{Visualizing alignments with \sft{esl-ssudraw}}

SSU rRNA alignments are large and difficult to view in a meaningful
way. The \prog{esl-ssudraw} program introduced for visualizing masks
earlier in this tutorial can be also used to display statistics of
a particular alignment on the consensus SSU secondary structure of the
model used to create the alignment. As before,
using \prog{esl-ssudraw} requires a template postscript file of the
consensus secondary structure. The template files for the 3 default
\textsc{ssu-align} version 0.1 seed models are included as
\prog{\$INFDIR/ssu-align-0.1/seeds/ss-diagrams/{archaea,bacteria,eukarya}-0p1.ps}. 
(Unfortunately, it is difficult to create new template files for
additional models you might build for your own analyses.) 

\prog{esl-ssudraw} can be run in two different modes. In the default
mode, \emph{alignment} mode, the structure diagrams will display
statistics on the alignment. In \emph{individual} mode, the structure
diagrams will show individual sequences in the alignment by displaying
the actual residues at each consensus position of the alignment. Note
that the diagrams are always of the consensus model defined in the
template file. In alignment mode, only statistics of consensus
positions are displayed. In individual mode, only residues that align to
consensus positions are displayed.

The following table summarizes the different statistics that can be
created in alignment mode.
An example of each of these types of
diagrams is included as the indicated figure for the default
eukaryotic seed alignment. The postscript and pdf files for each of
these included figures, as well as analogous diagrams for the
archaeal and bacterial seeds are included in 
\prog{\$INFDIR/ssu-align-0.1/seeds/ss-diagrams/}, named as indicated 
in the ``file'' column of the table. 

\begin{center}
\begin{tabular}{llll} \hline
\prog{esl-ssudraw} option(s) & statistic                     &  figure & file \\ \hline
\prog{<none>}                & information content           & \ref{fig:eukinfo} & \prog{eukarya-0p1-info} \\
& & & \\
\prog{-q --prob}                & average posterior probability & \ref{fig:eukprob} & \prog{eukarya-0p1-prob} \\
& & & \\
\prog{-q --ins}                 & frequency of insertions       & \ref{fig:eukins}   & \prog{eukarya-0p1-ins} \\
                             & after each position           & & \\
& & & \\
\prog{-q --dall}                & frequency of deletions        & \ref{fig:eukdall}  & \prog{eukarya-0p1-dall} \\
& & & \\
\prog{-q --dint}                & frequency of internal deletions & \ref{fig:eukdint}  & \prog{eukarya-0p1-dint} \\
                             & (excluding terminal deletions)  & & \\
& & & \\
\prog{-q --struct}              & additional information from     & \ref{fig:eukstruct} & \prog{eukarya-0p1-struct} \\
                             & conserved structure \\
\end{tabular}
\end{center}

If more than one of these options are used, the program will create a
multi-page postscript document with each diagram on a separate page.
If the \prog{-q} option is omitted, the program will create the
information content diagram as the first page by default.
As an example of how these files were created, the command

\user{esl-ssudraw -q --dall ../eukarya-0p1.stk eukarya-0p1.ps eukarya-0p1-dall.ps}

was used to create Figure~\ref{fig:eukdall} while in the 
\prog{\$INFDIR/ssu-align-0.1/seeds/ss-diagrams/} directory.

\newpage

\begin{figure}
\includegraphics[height=8.5in]{Figures/eukarya-0p1-info}
\label{fig:eukinfo}
\end{figure}

\newpage

\begin{figure}
\includegraphics[height=8.5in]{Figures/eukarya-0p1-prob}
\label{fig:eukinfo}
\end{figure}

\newpage

\begin{figure}
\includegraphics[height=8.5in]{Figures/eukarya-0p1-ins}
\label{fig:eukinfo}
\end{figure}

\newpage

\begin{figure}
\includegraphics[height=8.5in]{Figures/eukarya-0p1-dall}
\label{fig:eukinfo}
\end{figure}

\newpage

\begin{figure}
\includegraphics[height=8.5in]{Figures/eukarya-0p1-dint}
\label{fig:eukinfo}
\end{figure}

\newpage

\begin{figure}
\includegraphics[height=8.5in]{Figures/eukarya-0p1-struct}
\label{fig:eukinfo}
\end{figure}









\subsection{Creating a truncated model of a specific region of SSU rRNA}

Many SSU rRNA sequencing studies target a specific region of the SSU
rRNA molecule using PCR primers at the boundaries of that region. For
such studies it is recommended to build a new CM that only models the
region of the molecule targetted by the study. There are two reasons
for this. The first is speed; the running time of \textsc{ssu-align}
decreases as the model size it's using decreases. The second reason is
that aligning a SSU subsequence to a model of only the region that
subsequence is derived from, relative to a model of the entire SSU
molecule, should slightly increase alignment accuracy. This is because
the uncertainty of what region of the full molecule the subsequence
should align to is eliminated. In this section I'll demonstrate how to
create a CM of a specific region of SSU and use it to create
alignments. 

For this example imagine our study is only targetting bacterial SSU
rRNA. We will use the bacterial SSU seed alignment that is included
with \textsc{ssu-align} as a starting point for creating our new,
truncated CM. The first step is to determine what consensus positions
in the bacterial seed alignment's consensus structure the targetted
region corresponds to. The consensus structure of the bacterial seed
alignment is shown in the ``Models'' section on page X.

Create a new directory and copy the bacterial seed alignment in
\prog{seeds/bacteria-0p1.stk}, the default parameters
file \prog{sa-0p1.params}, and the file \prog{tutorial/partial.fa}
there.

Let's say our 5' primer begins at consensus position 35 and our 3'
primer ends at position 397.  In practice, you'll have to manually
find your primer site and determine their positions on this consensus
structure. On the structure diagrams in the ``Models'' section, every
hundredth residue is numbered, and every tenth residue is marked with
a tick mark, which should help you find the relevant positions.  If
you have a subsequence that exactly spans from one primer to another,
you can align it to the appropriate \textsc{ssu-align} model, then
number the consensus positions of that alignment with
\prog{esl-alimanip --num-rf} and examine the numbered alignment to
determine the primer positions.

\begin{srefaq}{If I'm creating a truncated model for sequences derived
    using specific primers, should I include the primer sequences
    within the new model or not?} You should include the primers
  because they will help the program correctly align each
  sequence. The primer sites are very highly conserved so they are
  simple for the program to correctly align and anchor the alignment
  of the rest of the sequence.
\end{srefaq}

For this example, the first step towards creating a truncated model is
to create a truncated seed alignment that only models between
consensus positions 35 and 397. We can do this with the
\prog{esl-alimanip} program using the full bacterial seed alignment as
input:

\user{esl-alimanip --start-rf 35 --end-rf 397 -o bac-35-397.stk bacteria-0p1.stk}

This command creates a new alignment called \prog{back-35-397.stk} which
includes the subset of the columns from \prog{bacteria-0p1.stk} that lie
between consensus positions 35 and 397 inclusively.

The next step is to build a new model from this new alignment with
\textsc{infernal} 1.0's \prog{cmbuild} program. If this program is in
your path, you can execute it with \prog{cmbuild}, otherwise you'll
need to provide the full path. We will specify the name we want
to give the model with the \prog{-n} flag:

\user{cmbuild -n bac-35-397 --enone --gapthresh 0.8 bac-35-397.cm bac-35-397.stk}

\begin{srefaq}{Why did you specify \prog{--enone} and
    \prog{--gapthresh 0.8} as command-line flags to \prog{cmbuild}?}
    These are the recommended ``best-practice'' options for building
    models for SSU alignment. The \prog{--enone} flag tells the
    program to turn off entropy-weighting, a parameterization
    technique used to make CM homology search more sensitive
    \cite{Nawrocki07} but that seems slightly detrimental to CM
    alignment accuracy with SSU rRNA models. The \prog{--gapthresh
      0.8} flag tells the program to define any column that has less
    than 80\% gaps in the seed as a consensus column. Different values
    than 0.8 could be used here, but 0.8 empirically seems to yield
    good performance for SSU alignment. The \prog{--enone} and
    \prog{--gapthresh 0.8} flags were used to build
    \textsc{ssu-align}'s five default models in the \prog{seeds/}
    subdirectory of the package.
\end{srefaq}

Now you can begin using your new model \prog{bac-35-397.cm} to align
SSU sequences. You have two options.  You can either use your new
model by itself as the only model in an \prog{ssu-align} run, or you
can combine it with other models to create a multi-model file to use
with \prog{ssu-align}. The former option is recommended if you expect
all of your sequences to match the truncated model, e.g. in this case,
be bacterial SSU subsequences that map close to the 35-397
region. I'll run through an example of this below. The latter option,
combining this model with others, is recommended if only a subset of
the sequences you will analyze are expected to match the truncated
model. In that case, the other models you combine the truncated model
should span the diversity of the other sequences you expect in your
sequence dataset. (An example of using a multi-model file with
\textsc{ssu-align} is demonstrated in the basic tutorial section).

Imagine we expect all the sequences in our sequence dataset are
bacterial sequences that match near the 35-397 region. An example
sequence file with 8 such sequences derived from larger sequences in
the \prog{rocks.fa} file used in the basic tutorial is included in
\prog{partial.fa}. I artificially added 10 random residues to the 5\' and 3\'
ends of the 8th sequence to demonstrate that the program can 
trim residues it deems nonhomologous to the model from the ends of
the sequences before alignment.

\user{ssu-align bac-35-397.cm partial.fa single sa-0p1.params}

The program takes about 2 seconds to run. 

Take a look at the \prog{single.scores} file in the \prog{single/}
subdirectory:

\begin{sreoutput}
#                                                      best-matching model                 
#                                       -------------------------------------------------  
#     idx  sequence name                model name   beg   end    CM sc   struct   HMM sc
# -------  ---------------------------  ----------  ----  ----  -------  -------  -------
        1  gi|146141790|gb|EF522294.1|  bac-35-397     1   344   403.20   100.11   312.76
        2  gi|146141797|gb|EF522301.1|  bac-35-397     1   348   463.37    69.69   402.70
        3  gi|146141805|gb|EF522309.1|  bac-35-397     1   315   475.02    63.31   416.05
        4  gi|146141812|gb|EF522316.1|  bac-35-397     1   339   480.57    68.35   423.81
        5  gi|146141828|gb|EF522332.1|  bac-35-397     1   336   450.46    70.82   392.66
        6  gi|146141831|gb|EF522335.1|  bac-35-397     1   332   380.82    94.56   297.11
        7  gi|146141832|gb|EF522336.1|  bac-35-397     1   334   476.07    69.57   415.65
        8  gi|146141837|gb|EF522341.1|  bac-35-397    11   363   338.11    83.56   285.87
\end{sreoutput}

Note how the alignment of the final sequence begins at position 11 and
ends at 363. The program truncated the first and last 10 nonhomologous
residues that I had manually added (that sequence is 373 residues long
in \prog{partial.fa}).


\subsection{Splitting up large datasets to run in parallel on a cluster}

If you have access to a compute cluster you can partition your input
dataset and run \textsc{ssu-align} in parallel on multiple machines. A
motivated user could certainly write their own scripts to do this, but
the \prog{ssu-align} program has options to facilitate this
parallelization so you don't have to.  It also will create a script
that will merge the separate alignments created for each partition
into a single master alignment.

In this section we'll walk through an example of how to do this for a
small dataset.  The sequence file \prog{tutorial/seed-30.fa} includes
30 randomly chosen sequences from the complete set of seed sequences
from the 5 default models of \textsc{ssu-align} (there are 205 total
seed sequences, as shown in the table of seed alignment statistics in
section~\ref{section:chap9}). To follow this example, create a new
directory and copy \prog{ssu-align-0.1/tutorial/seed-50.fa}, the file
\\ \prog{ssu-align-0.1/tutorial/janelia-sa-0p1.params} and the default parameters file
\\ \prog{ssu-align-0.1/sa-0p1.params} there.

There are two ways \textsc{ssu-align} can split up a job into multiple
jobs to run in parallel. You can either specify the number of jobs
(\prog{<x>}) to create, or the number of sequences (\prog{<y>}) you
want each job to handle. These are invoked using the command-line
options \prog{-c <x>} and \prog{-n <y>} respectively.  In this example
we'll use the \prog{-c} method.

One important modification should be made to the parameters file
before you use \textsc{ssu-align} to split up a large job to run in
parallel on a cluster. You'll want to create a new parameters file
(you can use the default \prog{sa-0p1.params} as a starting point) and
add two lines to it defining a prefix string and suffix string that
should appear before and after the \prog{ssu-align} call if you were
submitting it as a job for the cluster to your local job
scheduler. For example, look at the final 3 lines of the file
\prog{tutorial/janelia-sa-0p1.params}:

\begin{sreoutput}
$cluster_prefix = "qsub -N ssualign -o ssualign.out -b y -cwd -V -j y '";
$cluster_suffix = "'";
1;
\end{sreoutput}

The definitions of the variables \prog{\$cluster\_prefix} and
\prog{\$cluster\_suffix} tell \prog{ssu-align} what prefix and suffix
strings, respectively, to add to the \prog{ssu-align} runs it will
create for each partition. These specific strings correspond to the
format used by the SGE (Sun Grid Engine) \prog{qsub} program we use
for scheduling jobs on our cluster here at Janelia Farm in Virginia.
Importantly, you'll need to change these to the format required by
your own compute resources.

If you want the jobs to run in parallel on your local machine, you
would replace these three lines with:
\begin{sreoutput}
$cluster_prefix = "";
$cluster_suffix = " &";
1;
\end{sreoutput}

Because our example dataset here is only 30 sequences we really don't
need a cluster to run it, so we don't need to define the
\prog{\$cluster\_prefix} and \prog{\$cluster\_suffix} variables in our
parameters file. We'll use the default \prog{sa-0p1.params}.

Let's say we want to split up our 30 sequences into 5 separate files
of 6 sequences each and run each set of 6 independently. 
To execute the script:

\user{ssu-align -c 5 ssu3-0p1.cm seed-30.fa seed-30-c5 sa-0p1.params}

The program finishes in about 1 second. It will print information on
the files it has created:

\begin{comment}
# ssu-align :: define and align SSU rRNA sequences
# SSU-ALIGN 0.1 (June 2009)
# Copyright (C) 2009 HHMI Janelia Farm Research Campus
# Freely distributed under the GNU General Public License (GPLv3)
# - - - - - - - - - - - - - - - - - - - - - - - - - - - - - - - - - - - -
# command: ssu-align -c 5 ssu3-0p1.cm seed-30.fa seed-30-c5 sa-0p1.params
# date:    Wed Aug 19 18:57:15 2009
#
\end{comment}
\begin{sreoutput}
# Prep mode: Splitting up ssu-align job into 5 smaller jobs.
#
# output file name     description                                                 
# -------------------  ------------------------------------------------------------
  seed-30.fa.1         partition 1 fasta sequence file
  seed-30.fa.2         partition 2 fasta sequence file
  seed-30.fa.3         partition 3 fasta sequence file
  seed-30.fa.4         partition 4 fasta sequence file
  seed-30.fa.5         partition 5 fasta sequence file
  seed-30-c5.sh        shell script that will run ssu-align 5 times
  seed-30-c5.merge.pl  perl script to merge alignments when seed-30-c5.sh completes
  seed-30-c5.log       log file (*this* text printed to stdout)
#
# All output files created in directory ./seed-30-c5/
\end{sreoutput}

The first five fasta files are partitions of the original sequence
file \prog{seed-30.fa}. Each has 6 sequences in it. The file
\prog{seed-30-c5.sh} is a shell script that will execute
\prog{ssu-align} five times, once for each of the partitions. 
If we had defined \prog{\$cluster\_prefix} and
\prog{\$cluster\_suffix} this script would include those strings to
allow submission to a cluster. Since we did not our script will simply
execute \prog{ssu-align} three times in succession. The file
\prog{seed-30-c5.merge.pl} is a perl script we can use to merge the
individual alignments created by each of the five jobs together.
The \prog{seed-30-c5.log} file contains the exact text that was just
printed to the screen. All of these files were created in a new
directory called \prog{seed-30-c5}.

The next step is to run the \prog{seed-30-c5.sh} script, which will
execute \prog{ssu-align} three times. First, enter the
\prog{seed-30-c5/} directory, then type:

\prog{sh seed-30-c5.sh}

The output of the five \prog{ssu-align} runs will begin to print to
the screen. When one run finishes, the next will immediately
start. Each run should take about 20 seconds, so all five will take a
couple of minutes. 

Once they're done we can move to the next step: merging
the alignments. If we had defined \\ \prog{\$cluster\_prefix} and
\prog{\$cluster\_suffix}, each line of our script would have submitted
a separate job to our cluster. The script would finish quickly, but
the jobs would still be running. \textbf{\emph{Important: }} you'll
need to wait until all the jobs are finished running on the
cluster before attempting to merge the alignments.

Continuing with our example, to merge the alignments:

\user{perl seed-30-c5.merge.pl}

This script will create a merged alignment for each model in the
original cm file (\prog{ssu3-0p1.cm} in this case) that was the
best-matching model for at least 1 sequence in any of the three runs.
For each such model, the script will merge two alignments at a time
using \textsc{infernal}'s \prog{cmalign} program (see the next section
on ``Merging multiple alignments together'' for an example)
until a single alignment with all the sequences is created. 

For example, in this case the alignment from jobs 1 and 2 would be
merged first, then the alignments from jobs 3 and 4 are merged. Now
three alignments exist, one from jobs 1+2, one from 3+4 and one from
5. Next 1+2 are merged with 3+4, and finally the resulting alignment
of 1+2+3+4 is merged with the alignment from job 5. If you look at the
code in \prog{seed-30-c5.merge.pl} it may look messy, but this
simple binary merging is all that it is doing. The intermediate
alignments are deleted once they are no longer needed as the script
proceeds. If you have a large number of alignments to merge and/or a
large number of total sequences this script may take a long time. 

In this case, 3 merged alignments are created, as reported by the
script:

\begin{sreoutput}
Merged alignment for archaea  CM saved to seed-30-c5.1-5.archaea.merged.stk
Merged alignment for bacteria CM saved to seed-30-c5.1-5.bacteria.merged.stk
Merged alignment for eukarya  CM saved to seed-30-c5.1-5.eukarya.merged.stk
\end{sreoutput}

These alignments will be 100\% identical to the alignments that would
have been created if we had not split up this job, but rather aligned
all 30 sequences with a single run of \prog{ssu-align}.



\subsection{Merging multiple alignments together}

The \prog{cmalign} program of \textsc{infernal} is capable of merging
two alignments into one. The two alignments must have both been created by
\prog{cmalign} (and the same version of \prog{cmalign}, 1.0 or later),
and must have been created using the same exact CM. This ability is
potentially useful for saving time when aligning a large number of
sequences if you have access to a compute cluster, as described in the
previous section (``Splitting up large alignment jobs''), or if you
want to merge an existing reference alignment with a newly created
one. Combining a new alignment with a reference may be useful in
downstream phylogenetic analysis if you know the classification of the
sequences in the reference alignment.

Imagine we wanted to merge the archaeal sequences from the
\prog{rocks.fa} sequence dataset from the basic tutorial with 
an alignment of the archaeal seed sequences. 

Merging alignments only makes sense and saves time if you've already
computed the two alignments you want to merge. For this example I've
provided the two alignments in the \prog{/tutorial} subdirectory:

\begin{description}
\item[\emprog{rocks.archaea.stk}]
  An alignment of the archaeal sequences from the \prog{rocks.fa}
  dataset. The basic tutorial steps through how to create this file.

\item[\emprog{seed.archaea.stk}]
  An alignment of the archaeal seed sequences, created with the
  command \prog{ssu-align ../seeds/archaea-0p1.cm ../seeds/archaea-0p1.fa seed ../ssualign.0p1.params}
  (Note: this file is NOT identical to the seed alignment
  \prog{archaea-0p1.stk}. I've realigned those sequences to the
  archaeal model to obtain posterior probability confidence estimates
  in the alignment. 
\end{description}

If \prog{cmalign} version 1.0 is in your path, you 
can merge these two alignments into a single alignment called
\prog{seed-rocks.archaea.stk} with

\user{cmalign --merge -o seed-rocks.archaea.stk ../seeds/archaea-0p1.cm seed.archaea.stk rocks.archaea.stk}

The resulting alignment will be 100\% identical to the alignment
\prog{cmalign} would have created if it were used to align a single
sequence file with the seed sequences and rocks sequences together.








