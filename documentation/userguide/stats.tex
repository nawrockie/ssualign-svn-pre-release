\section{Running times and output file sizes for example datasets}
\label{sec:stats}

This section contains timing and output file size statistics
for aligning, masking and drawing SSU sequences with the
\sft{SSU-ALIGN} package to give you an idea of how much time and disk
space you'll need for processing various sized datasets. The
timing statistics reported here are for single execution threads
running on 3.0 Intel Xeon processors. All programs were run with
default parameters (no options), with the exception that some of the
synthetic-v4 datasets were processed with a truncated set of models as
explained further below. The statistics are displayed in 
Tables~\ref{tbl:sizes}, \ref{tbl:ttimes}, and \ref{tbl:ptimes}.

\input{tbl-stats.tex}

Statistics are reported for real and synthetic datasets. The real
datasets were obtained exclusively from existing SSU rRNA databases:
namely \sft{greengenes} \cite{DeSantis06a}, \sft{silva}
\cite{Pruesse07} and \sft{RDP} \cite{Cole09}. For \sft{greengenes},
two separate datasets were processed: the \emph{core set}, which is the reference
alignment used by that database (``Ggenes-core'' dataset), and all the
SSU sequences in the database (``Ggenes-full'' dataset) as after the
March 23, 2010 update. Two datasets from \sft{silva} were
processed: the \emph{Ref} dataset and the \emph{Parc} dataset
(``Silva-Ref'' and ``Silva-Parc'' datasets in the tables). The
\emph{Ref} set is a subset of \emph{Parc}, the \emph{Ref} sequences
are generally longer and higher quality than the other sequences in
\emph{Parc}. Both sets of sequences are from release 102 of the
database. Finally, a single \sft{RDP} dataset was analyzed: the full
set of sequences in release 10, update 20.

The synthetic datasets were fabricated using the \sft{infernal}
package. Specifically the \prog{cmemit} program was used with default
settings and the \prog{-n <N>} option to generate \prog{<N>}
sequences, where \prog{<N>} is reported in the \prog{number of seqs}
columns of the tables. For the ``synthetic-full'' datasets, the
default three \sft{ssu-align} SSU CMs were used to generate the
sequences. For the ``synthetic-v4'' datasets, smaller, truncated
versions of the default archaeal and bacterial models were used to
generate the sequences. These CMs only model the V4 hypervariable
region of SSU. They were constructed using the same \prog{ssu-build}
commands from the tutorial section of this guide (section
~\ref{sec:tutorial-build-v4}).

For all datasets the default models were used to align, mask and draw
sequences using \prog{ssu-align}, \prog{ssu-mask} and
\prog{ssu-draw}. These runs with default settings are summarized in the
rows of the tables that include ``default'' in the ``models'' column.
Additionally, for the ``synthetic-v4'' dataset, the smaller V4
models were used for aligning and masking. Drawing was impossible for
these runs because non-default models were used. These runs correspond
to the rows in the tables marked ``v4'' under ``model''.

For all datasets with more than $10,000$ sequences \prog{ssu-prep} was
used to parallelize alignment. These datasets were split up into $100$
separate jobs and run in parallel on a cluster. This is indicated by
the ``number of procs'' column in tables~\ref{tbl:ttimes} and
~\ref{tbl:ptimes}. No such column exists in table ~\ref{tbl:sizes} because
parallelization does not affect the sizes of the final output files.

As an example of a non-parallel run, the actual commands used to process the
\sft{greengenes} core set dataset were as follows. The FASTA file
included the unaligned core set sequences is named \prog{ggcs-unaln.fa}. The
following three commands would align, mask and draw the core
set, respectively. All output files will be created in a directory called
\prog{ggcs}. Importantly, the three commands  must be run in succession
(masking cannot begin until alignment is complete). 

\user{ssu-align ggcs-unaln.fa ggcs}

\user{ssu-mask ggcs}

\user{ssu-draw ggcs}

As an example of a parallel run, the actual commands used to process the
\sft{greengenes} full dataset were as follows. The FASTA file
included the unaligned core set sequences is named \prog{gg-unaln.fa}. The
following four commands would partition, align, mask and draw the full
dataset, respectively. All output files will be created in a directory
called \prog{gg}. Importantly, the four commands must be run in succession
(masking cannot begin until alignment is complete). 

\user{ssu-prep gg-unaln.fa gg 100 janelia-cluster-presuf.txt}

\user{sh gg.ssu-align.sh}

\user{ssu-mask gg}

\user{ssu-draw gg}

The \prog{ssu-prep} command above creates the
\prog{gg.ssu-align.sh} script that will submit 100 \prog{ssu-align}
jobs to the cluster here at Janelia. The Janelia-specific prefixes and
suffixes for the commands are in the file
\prog{janelia-cluster-presuf.txt} which is in the \prog{tutorial/}
subdirectory of \sft{ssu-align}. See the tutorial section
~\ref{sec:tutorial-prep}.

The statistics in tables~\ref{tbl:sizes}, \ref{tbl:ttimes} and
\ref{tbl:ptimes} highlight some important considerations:

\begin{itemize}

\item Deep alignments are mostly composed of insert columns which
  dramatically inflate the size of the alignment files
  (table~\ref{tbl:sizes}). All insert columns will be removed
  automatically by \prog{ssu-mask} (see section~\ref{sec:background}
  for more discussion of insert versus consensus columns). Note that
  insert columns will not be included in output alignments of
  \prog{ssu-align} when the \prog{--rfonly} option is used.

%\item Alignment is by far the slowest step in the pipeline
%  (table~\ref{tbl:ttimes}). Prep'ing, masking, drawing and merging are
%  all much faster than alignment, but only the \prog{ssu-align} step
%  can be parallelized. 

\item Full-length SSU sequences are aligned at about 1
  sequence/second (table~\ref{ptimes}).

\item Shorter ($\sim$240 nt) V4 subsequences are aligned at about 10
  sequences per second with the default, full-length models (table~\ref{ptimes}).

\item Aligning short sequences with truncated models is significantly
  faster than with full models. V4 subsequences are aligned at about
  100 sequences per second with V4-specific models
  (table~\ref{ptimes}).


\item Aligning 1 million typical SSU sequences requires about 300
  CPU hours. This can be parallelized on a cluster of 100 processors
  in about 3.5 hours. Note that for about 25 of the final 30
  minutes only one processor is still running, this is the processor
  that is performing the merge of the 100 per-job alignments. 

\end{itemize}
%The column headings are:
%
%\begin{sreitems}{}
%\item[\prog{dataset}] name of the dataset, as explained in the text above.

%\item[\prog{models}] set of models used, either \prog{default} or
%  \prog{v4}, the latter being the V4 region of archaeal and
%  bacterial SSU, created in section \ref{sec:tutorial-build-v4} of
%  the tutorial.

%\item[\prog{domain}] which domain the row pertains to.%
%
%\item[\prog{domain}] which domain the row pertains to.%

%\end{sreitems}
